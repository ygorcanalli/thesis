%% This is file `example.tex',

%% generated with the docstrip utility.
%%
%% The original source files were:
%%
%% coppe.dtx  (with options: `example')
%% 
%% This is a sample monograph which illustrates the use of `coppe' document
%% class and `coppe-unsrt' BibTeX style.
%% 
%% \CheckSum{1620}
%% \CharacterTable
%%  {Upper-case    \A\B\C\D\E\F\G\H\I\J\K\L\M\N\O\P\Q\R\S\T\U\V\W\X\Y\Z
%%   Lower-case    \a\b\c\d\e\f\g\h\i\j\k\l\m\n\o\p\q\r\s\t\u\v\w\x\y\z
%%   Digits        \0\1\2\3\4\5\6\7\8\9
%%   Exclamation   \!     Double quote  \"     Hash (number) \#
%%   Dollar        \$     Percent       \%     Ampersand     \&
%%   Acute accent  \'     Left paren    \(     Right paren   \)
%%   Asterisk      \*     Plus          \+     Comma         \,
%%   Minus         \-     Point         \.     Solidus       \/
%%   Colon         \:     Semicolon     \;     Less than     \<
%%   Equals        \=     Greater than  \>     Question mark \?
%%   Commercial at \@     Left bracket  \[     Backslash     \\
%%   Right bracket \]     Circumflex    \^     Underscore    \_
%%   Grave accent  \`     Left brace    \{     Vertical bar  \|
%%   Right brace   \}     Tilde         \~}
%%
\documentclass[dsc,english]{coppe}
\usepackage{natbib}
\usepackage{amsmath,amssymb}
\usepackage{hyperref}
\usepackage{amssymb}
\usepackage{tikz}
\usepackage{multirow}
\usepackage{amsmath}
\usepackage{booktabs} 
\usepackage{subcaption}
\usepackage{amsthm}
\usepackage{url}
\usepackage{lineno}

\newtheorem{definition}{Definition}
\newtheorem{theorem}{Theorem}
\DeclareMathOperator*{\argmin}{arg\,min}
\DeclareMathOperator{\FTL}{FTL}
\DeclareMathOperator{\rank}{rank}



\linenumbers % Enable line numbering
\modulolinenumbers[5] % Set line numbering interval


\makelosymbols
\makeloabbreviations

\begin{document}
  \title{Função de custo e penalização robustas para aprendizado de máquina justo}
  \foreigntitle{Robust loss and penalty to fair machine learning}
  \author{Ygor}{de Mello Canalli}
  \advisor{Prof.}{Geraldo}{Zimbrão da Silva}{D.Sc.}
  \advisor{Prof.}{Filipe}{Braida do Carmo}{D.Sc.}

  \examiner{Prof.}{Geraldo Zimbrão da Silva}{D.Sc.}
  \examiner{Prof.}{Filipe Braida do Carmo}{D.Sc.}
  \examiner{Prof.}{Geraldo Bonorino Xexéo}{D.Sc.}
  \examiner{Prof.}{Daniel Sadoc Menasché}{Ph.D}
  \examiner{Prof.}{Carlos Eduardo Ribeiro de Mello}{Ph.D.}
  \department{PESC}
  \date{08}{2024}

  \keyword{Fair machine learning}
  \keyword{Label noise robustness}
  \keyword{Multi-objective optimization}
  \keyword{Redlining effect}

  \maketitle

  \frontmatter
  \dedication{A algu\'em cujo valor \'e digno desta dedicat\'oria.}

  \chapter*{Agradecimentos}

  Gostaria de agradecer a todos.

  \begin{abstract}

  problema e importância do problema, proposta, metodologia experimental, resumo dos resultados, contribuições e legado no estado da arte.

  \end{abstract}

  \begin{foreignabstract}

  In this work, we present ...

  \end{foreignabstract}

  \tableofcontents
  \listoffigures
  \listoftables
  \printlosymbols
  \printloabbreviations

  \mainmatter
  
  \chapter{Introduction}

Brisha Borden was running late to pick up her god-sister from school. She and her friend, both 18-year-old girls, took an unlocked kid’s bicycle and scooter and ride for a while. They were arrested and charged with theft. Meanwhile, Vernon Prater, a seasoned criminal with prior history of armed robbery and years in prison, was caught by theft tools. A computer program predicted their future criminal behavior: Borden, a black woman, was rated high risk; Prater, a white man, was rated low risk. However, the algorithm got it wrong. Borden remained crime-free, while Prater was again serving an eight-year prison term for a subsequent theft.

This famous case revealed by a ProPublica's investigative reporting evidenced the biased behavior of Correctional Offender Management Profiling for Alternative Sanctions (COMPAS), a criminal risk assessment tool owned by Northpointe and used by Florida's criminal justice system~\citep{machine_bias}. Examples like that are not extraordinary, decision making systems are incurring in bias by reproducing prejudices and disparities from society all around the world. Although the machine learning bias can not be resolved only by computing science and statistics solutions, we delve into this issue in order to provide tools for decision makers, civil society and other stakeholders to tackle this humongous problem. 

\section{Contextualization}

The issue of fairness in machine learning has recently risen to prominence due to its implications in real-world decision-making systems~\citep{Mehrabi2019,Hutchinson2019}. Addressing biases and discrimination is a relevant frontier in decision-making systems, as equitable outcomes across various demographic groups is both an ethical imperative and often a legal requirement. Though fairness is a multifaceted concept, it has been deeply examined within the context of machine learning. The literature presents a variety of fairness definitions, drawing concepts from political philosophy and computational techniques~\citep{caton2023, Hutchinson2019}. Choosing an equitable machine learning model requires the selection of a fitting definition of fairness, tailored to the specific problem at hand. Many such definitions can be precisely articulated, allowing models to be evaluated based on their predictions.

One inherent challenge in fair machine learning is the balance between fairness and predictive performance. Efforts to mitigate unfairness often compromise the model's predictive performance, a trade-off that has been well documented~\citep{Mehrabi2019, caton2023}. Predictors that are less biased against marginalized groups may deviate from the true class, resulting in sub-optimal performance. Also, introducing fairness considerations adds constraints to the model, further complicating the optimization process~\citep{Zafar2017b}. 

In this context sounds intuitive the idea of just removing protected attributes such as race, sex and age, so that the model would be unable to produce outcomes using these information in a biased way. Unfortunately this approach not only does not work, as the model will probably infer these characteristics using proxy features available on data, but also it can accentuate disparities due lack of information. This tricky phenomena is know as Redlining Effect~~\citep{Pedreschi2008} and was extensively documented~~\citep{Mehrabi2019,caton2023,Hort2023}.

\section{Objectives}

Although noise and unfairness in the context of classification are not identical phenomena, there are relevant similarities, especially in their effects on data. Both changes the appropriate or expected values of an instance's attributes or label. While noise corrupts the observable value of an attribute or class without affecting the true value, unfairness acts directly on the social reality that generates these values, corrupting both the observable and true values. Considering the succeeds approaches to produce label noise robustness through loss correction~\citep{Patrini2017}, which uses different levels of uncertainty according observed label, we raise the following hypothesis:

\begin{hypothesis}\label{hyp:ftl}
A loss correction approach that considers different levels of corruption according individual's sensitive attribute and available class reduces unfairness. 
\end{hypothesis}

Thus, the main objective of this study is to produce a loss correction tailored to reconsider the predicted outcomes of a binary classifier taking into account different levels of unfairness according individual's social group and available class, addressing Hypothesis~\ref{hyp:ftl}.

Another relevant topic tackled by this work is the Redlining Effect, which perversely penalizes individuals of socially marginalized groups, even on well-intentioned approaches to mitigate unfairness. Intuitively, reducing model's reliance on those sensitive feature's proxies would promote fairness. Thus, in order to achieve a Redlining Effect robustness, we raise the following hypothesis:

\begin{hypothesis}\label{hyp:rpr}
A regularization approach that penalizes model's dependency on sensitive attribute proxies reduces the effects of redlining an thus promoting fairness. 
\end{hypothesis}

Therefore, to accomplish Hypothesis~\ref{hyp:rpr} this study pursue this secondary objective of developing a regularization approach capable of properly identifying and penalizing the proxies of sensitive feature. 

\section{Contributions}

In light of referred challenges, we introduce the Fair Transition Loss, a novel approach to fair classification. This method estimates the influence of historical and societal biases on outcome probabilities for distinct groups within dataset. For instance, individuals from marginalized groups might have lower chances of favorable outcomes compared to their counterparts from privileged groups. Such disparate probabilities can be represented by transition matrices. Drawing inspiration from label noise robustness, we incorporate these transition matrices information into the loss function to promote fairness. The proposed method has some hyperarameters, chosen by a Multi-Objective Optimization approach combining both fairness and model performance with a linear objective. This objective function is defined in such a way that the proposed approach is suitable to optimize a wide range of fairness and performance metrics.

The primary contribution of this study is the conceptualization of the Fair Transition Loss, a novel loss function influenced by label noise methodologies. The novelty of this work lies in applying label noise techniques directly within the model to mitigate unfairness. As far as we know, this is the first work to adopt label noise techniques directly to address fairness in machine learning. Additionally, the proposed method achieves state-of-art results in benchmark tests across common fair classification tasks. Our empirical results demonstrate that this method consistently outperforms many leading fair classification techniques in a variety of scenarios. This conceptualization and results are also published at \cite{Canalli2024}.

Although we draw some comparison between unfairness and noise concepts, delineating similarities and differences, it is important to emphasizes that this works does not describes bias and discrimination issues as a noise phenomena. Instead, we use some transition-matrix-based robust loss function from label noise literature in order to create a loss correction approach suitable to fair classification problems. The core concept here is to adjust the individual's probabilities according social group (i.e. protected and privileged) achieving the favorable or unfavorable outcome on a binary classification problem, which is attained trough a custom loss function considering the unfairness information from data in a transition matrix format.

Another contribution of the present study is a novel regularization approach to fair classification named Redlining Penalty Regularization, which uses feature's Chatterjee's xi correlation~~\citep{chatterjee2020new}  to the sensitive attribute in order to proportionately penalize model's dependency on it. Our empirical evaluation demonstrate that this approach effectively mitigate unfairness while keeping predictive performance, with benefits corresponding to redlining level on dataset. Also, we demonstrate that this approach can reduce bias while keeping predictive performance on a variety of scenarios, enhancing performance-fairness trade-off on both standard neural networks and those trained with Fair Transition Loss. To the best of our knowledge this is the first regularization approach to penalize model's dependency proportionately to feature's correlation to protected attribute in order to mitigate unfairness. 

\section{Results summary}

Our empirical evaluation compares the proposed methods with its counterparts on the fair classification benchmark datasets \textit{Adult Income}~\citep{misc_adult_2}, \textit{Bank Market}~\citep{misc_bank_marketing_222}, \textit{COMPAS Recidivism}~\citep{misc_compas}, and \textit{German Credit}~\citep{misc_statlog_(german_credit_data)_144} under multiple optimization objectives, each of them targeting maximize a predictive performance metric while minimizing a fairness metric. In every comparison we optimize all methods through an extensive hyperparameter tuning process, in order to guarantee that each one has the same competitive conditions. After the hyperparameter tuning phase the method is than retrained with the best hyperparameters and evaluated using a test set not used on tuning, whose metrics are reported in results. This complete process is then repeated over multiple runs through dataset resampling and an Almost Stochastic Order~\citep{dror2019deep} comparison is performed, which is a significance test suitable to compare complex machine learning models under hyperparameter optimization. The comparative studies are conducted under $4$ datasets to $6$ optimization objectives, which lead us to $24$ evaluation scenarios each. A complete description of these experimental evaluations are available at Section~\ref{sec:ftl_experimental} and Section~\ref{sec:rpr_experimental}, while the computational implementations, results and analysis are available at \cite{canalli2024_zenodo} as supplementary material.

In order to properly attend to Hypothesis~\ref{hyp:ftl} we compare the proposed loss correction named Fair Transition Loss (Section~\ref{sec:ftl_proposal}) with a standard Multi-Layer Perceptron and with classical and state-of-art fair classification approaches. The experimental results available at Section~\ref{sec:ftl_results} can be summarized as follow:
\begin{itemize}
    \item The proposed model achieves the best result on most of evaluated scenarios;
    \item The proposed model is the only one with competitive results in all evaluated scenarios.
\end{itemize}

Also, to attend to Hypothesis~\ref{hyp:rpr} we compare the proposed regularization approach named Redlining Penalty Regularization (Section~\ref{sec:rpr_proposal}) applied on a standard Multi-Layer Perceptron and on a Multi-Layer Perceptron with Fair Transition Loss. The experimental results available at Section~\ref{sec:rpr_results} can be summarized as follow:
\begin{itemize}
    \item The proposed regularization approach enhances the results on most of evaluated scenarios;
    \item The proposed regularization approach performs according the redlining level of the dataset, the higher the redlining effect higher the improvement.
\end{itemize}

\section{Thesis structure}

The remainder of this Thesis is structured as follows: Chapter~\ref{chap:fairness} delves into Fair Machine Learning describing core concepts, some related research topics, sources and types of algorithmic unfairness, definitions, metrics and fair classification approaches; Chapter~\ref{chap:ftl}, describes and evaluates Fair Transition Loss and its underlying principles, delineating related works encompassing multi-objective optimization in fair machine learning, classifications in the presence of label noise and studies bridging fairness and label noise. Chapter~\ref{chap:rpr}, describes and evaluates Redlining Penalty Regularizer and its fundamentals, describing some correlation coefficients suitable to this use and related works that applies a regularization approach to promote fairness; Chapter~\ref{chap:conclusions} presents conclusions drawn from our study, fashioning some considerations on the proposed methods, delineating the contributions and some insights to further research.
  
\chapter{Fair machine learning}

The field of Machine Learning (ML) has experienced significant growth and is increasingly applied in various societal domains such as healthcare, finance, and criminal justice. This growth raises important ethical and operational concerns, particularly regarding the principles of Fairness, Accountability, and Transparency (FAT) \citep{Memarian2023}. As ML algorithms increasingly influence a wide array of societal domains, including criminal justice, healthcare, finance, and employment, the requirement to ensure these systems are designed and implemented responsibly has become paramount. This section aims to delineate the significance, scope, and prevailing challenges associated with integrating FAT principles into ML, providing a foundation for the subsequent discussion.

Fairness in ML concerns the equitable and just treatment of all individuals, particularly those from historically marginalized or disadvantaged groups \citep{Mehrabi2019, caton2023}. It seeks to ensure that ML algorithms do not perpetuate existing biases or create new forms of discrimination. However, the multifaceted nature of fairness, encompassing various definitions and metrics, poses substantial challenges in operationalizing it within algorithmic frameworks. Further in this section we will explore these complexities, examining different conceptions of fairness and the inherent trade-offs they entail.

Accountability in ML pertains to the obligation of designers, developers, and maintainers of ML systems to be answerable for the outcomes of these systems \citep{Hutchinson2021}. It involves establishing mechanisms that allow for the tracing of decisions back to the entities responsible for the deployment of the ML algorithms. Accountability also encompasses the adherence to ethical standards, legal requirements, and societal norms. This discussion frequently involves mechanisms and practices that can promote accountability in ML, like auditing, documentation, and regulatory compliance.

Transparency, the third pillar, refers to the clarity and openness with which ML systems operate \citep{Burkart2021}. It involves the ability of stakeholders, including end-users, regulators, and the broader public, to understand how ML systems make decisions. Transparency is mandatory property of any automated decision making system to achieve trustworthiness, facilitating informed consent, and enabling the scrutiny necessary to identify and rectify biases. However, achieving transparency, particularly with complex models, presents its own set of technical and ethical challenges. This research topic includes issues as the trade-off between explainability and model performance, and discussing emerging approaches to tackle interpretability without sacrificing effectiveness.

The triad of Fairness, Accountability, and Transparency (FAT) along with data privacy forms the cornerstone of Trustworty Artificial Intelligence (TwAI). These principles are pivotal in ensuring that AI systems are developed and deployed in a manner that respects human rights, promotes social well-being, and maintains public trust. While accountability ensures that entities behind AI systems can be held responsible for their outcomes, transparency allows stakeholders to produce and maintain environments where AI systems can be scrutinized, understood, and corrected, thereby aligning their functionality with societal norms and values.

In this context of Trustworthy AI the European Union's High-Level Expert Group on Artificial Intelligence has outlined seven key principles that aim to ensure that AI systems are designed and used in a way that is ethically sound and trustworthy~\citep{TwAI_Europe}. These principles are fundamental for developing and maintaning decision making systems that are beneficial and avoid unintended harm. The seven principles are as follows:

\begin{description}
    \item[Human agency and oversight] AI systems should empower human beings, allowing them to make informed decisions and fostering their fundamental rights. At the same time, proper oversight mechanisms need to be ensured, which can be achieved through human-in-the-loop, human-on-the-loop, and human-in-command approaches
    \item[Technical Robustness and safety] AI systems need to be resilient and secure. They need to be safe, ensuring a fall back plan in case something goes wrong, as well as being accurate, reliable and reproducible. That is the only way to ensure that also unintentional harm can be minimized and prevented.
    \item[Privacy and data governance] besides ensuring full respect for privacy and data protection, adequate data governance mechanisms must also be ensured, taking into account the quality and integrity of the data, and ensuring legitimized access to data.
    \item[Transparency] the data, system and AI business models should be transparent. Traceability mechanisms can help achieving this. Moreover, AI systems and their decisions should be explained in a manner adapted to the stakeholder concerned. Humans need to be aware that they are interacting with an AI system, and must be informed of the system’s capabilities and limitations.
    \item[Diversity, non-discrimination and fairness] Unfair bias must be avoided, as it could have multiple negative implications, from the marginalization of vulnerable groups, to the exacerbation of prejudice and discrimination. Fostering diversity, AI systems should be accessible to all, regardless of any disability, and involve relevant stakeholders throughout their entire life circle.
    \item[Societal and environmental well-being] AI systems should benefit all human beings, including future generations. It must hence be ensured that they are sustainable and environmentally friendly. Moreover, they should take into account the environment, including other living beings, and their social and societal impact should be carefully considered. 
    \item[Accountability] Mechanisms should be put in place to ensure responsibility and accountability for AI systems and their outcomes. Auditability, which enables the assessment of algorithms, data and design processes plays a key role therein, especially in critical applications. Moreover, adequate an accessible redress should be ensured.
\end{description}


Although there are many aspects to consider to an ethical automated decision system with social impacts, the present text will concentrate predominantly on the aspect of fairness and negative social bias. Fairness is not only desirable for the development of just and equitable technological solutions but an obligation for maintaining the trustworthiness and acceptability of AI systems in diverse societal contexts.

\section{Sources and types of algorithmic unfairness}

The comprehensive survey conducted by \citet{Mehrabi2019} elucidates the multitude of biases that can pervade artificial intelligence applications, potentially leading to unfair outcomes. This analysis categorizes the various sources of bias, illustrating the multifaceted ways in which such biases can infiltrate different stages of machine learning processes, ranging from the initial data collection phase to the final algorithmic processing. The following exposition provides a short delineation of these sources of bias. To a rich discussion on this topic - including references, examples and real cases where each source of bias can emerge - we recommend the reading of the original work. The discussion here is with the purpose of proper describing the complexity and multifaceted nature of unfairness in machine learning models.


\begin{description}
    \item[Historical Bias] This is the existing societal bias that reflects past and present inequalities and prejudices. Historical bias is present in the data even before any machine learning model has interacted with it, due to inherent social and cultural inequalities;
    
    \item[Representation Bias] Occurs when the data sample does not accurately represent the entire population or certain subgroups within it. This can lead to machine learning models that perform well on majority groups but poorly on underrepresented groups;

    \item[Measurement Bias] Arises when the data collected does not accurately measure the real-world constructs it purports to measure. This type of bias can occur due to flawed data collection instruments or processes that systematically misrepresent certain groups;
    
    \item[Evaluation Bias] This type of bias occurs during the performance evaluation of machine learning models, where the evaluation criteria or methods may favor one group over others, leading to biased assessments of model performance;

    \item[Aggregation Bias] Happens when incorrect assumptions are made about the homogeneity of groups within the data. Aggregation bias can lead to misleading conclusions if the differences within and between groups or subgroups are not properly accounted for;
    
    \item[Population Bias] Similar to representation bias, population bias occurs when statistics, demographics, representatives, and user characteristics are different in the user population represented in the dataset, leading to models that are not generalizable across different demographic groups;

    \item[Simpson’s Paradox] This is a statistical phenomenon where a trend appears in several different groups of data but disappears or reverses when these groups are combined;

    \item[Longitudinal Data Fallacy] Occurs when cross-sectional data is treated as longitudinal, leading to incorrect conclusions about data trends over time;

    \item[Sampling Bias] Introduced by non-random sampling procedures, where certain members of the intended population are less likely to be included in the sample than others, leading to skewed data that does not accurately represent the entire population;
    
    \item[Behavioral Bias] Arises from variations in user behavior that differ across different platforms or contexts, affecting the data's representation of real-world phenomena;

    \item[Content Production Bias] Results from differences in how content is generated by different groups, with structural, lexical, semantic, and syntactic differences, influencing the data available for machine learning models; 

    \item[Linking Bias] Occurs in networked data, where the connections between nodes can misrepresent the true attributes or behavior of the nodes;

    \item[Temporal Bias] Reflects changes in data characteristics over time, due changes in representation or behaviors, which may not be accounted for in static machine learning models;
    
    \item[Popularity Bias] Occurs when popular items are more likely to be recommended or rated highly, not necessarily because of their quality but simply because of their initial, higher visibility and these popularity metrics are subject of mapinupation;

    \item[Algorithmic Bias] Introduced by the algorithms themselves, when they add bias that was not present in the input data:

    \item[User Interaction Bias] Results from the way system design influences user behavior in biased ways. This source of bias can be influenced by other types or subtypes, such as Presentation and Ranking Biasese:
    \begin{description}
        \item[Presentation Bias] This bias occurs when the way information is presented influences the outcomes. In machine learning, this can manifest through the design of user interfaces or the manner in which data is displayed, affecting user decisions and interactions;

        \item[Ranking Bias] Arises when algorithms prioritize certain data points over others in ranked lists or search results, which can distort visibility and perpetuate certain preferences or discriminations;
    \end{description}
    
    \item[Social Bias] Social biases are the preconceived notions and stereotypes held by societies, where individual actions or contents are socially influenced. These biases often find their way into data through collective social behaviors and decisions, influencing the training data used for machine learning models;

    \item[Emergent Bias] Emerges during the operation of a system, particularly as a result of changes in population, cultural values, or societal knowledge in the data over time. This type of bias is dynamic and can occur even if the initial model was unbiased, due to changes in the underlying data or context;

    \item[Self-Selection Bias] Occurs when the individuals selected for a study or dataset have self-selected in some way, producing a sample that is not representative of the general population. This can skew results and make the data less generalizable;

    \item[Omitted Variable Bias] Happens when a model overlooks certain relevant variables that are correlated with both the independent and dependent variables. Omitting these variables can lead to incorrect inferences about correlations and effects;

    \item[Cause-Effect Bias] This bias is a misunderstanding in the determination of causation; it can occur when correlations are mistaken for causal relationships without proper justification through causal inference techniques;

    \item[Observer Bias] Introduced by the expectations or preconceptions of those collecting or processing data, which can influence the outcomes subconsciously.;
    
    \item[Funding Bias] Refers to the influence that the source of funding can have on the conduct of research or development of algorithms. This type of bias can lead to results that favor the interests of the funding source, consciously or unconsciously.
\end{description}

These biases can pervade various stages of machine learning, from data collection to model evaluation and deployment, highlighting the importance of understanding and mitigating bias to achieve fairness in AI systems. Furthermore, the presence of biases can lead to feedback loops that exacerbate these inequalities over time. When biased data influence the decisions made by an AI system, these decisions can then be used to generate more data, which, if used to retrain the model, may reinforce and even amplify the existing biases. This cycle can create a self-perpetuating loop, making initial biases more entrenched and difficult to correct. Addressing feedback loops is critical, as they can progressively deteriorate the fairness of the system, leading to increasingly skewed outcomes that are harder to rectify. Effective strategies to break these loops include rigorous monitoring of model decisions, regular updates to training datasets to ensure diversity and representativeness, and the implementation of mechanisms that can detect and correct for emerging biases

Having outlined the various sources of unfairness in machine learning, \citet{Mehrabi2019} also explores different types of discrimination that arise from these biases. Understanding these types of discrimination is pivotal as they elucidate how biases, whether direct, indirect, systemic, statistical, explainable, or unexplainable, can culminate in unfair outcomes. Each type of discrimination demonstrates a distinct pathway through which biases embedded in data or algorithms manifest in practices and decisions, thus potentially perpetuating unfairness in AI systems. This comprehensive analysis helps in identifying targeted strategies to mitigate these discriminatory effects and underscores the importance of developing automated decision systems that are both just and equitable.


\begin{description}
    \item[Direct Discrimination] occurs when outcomes are directly affected by sensitive attributes such as race, gender, or age. This type of discrimination happens explicitly and is frequently legally prohibited;
    
    \item[Indirect Discrimination] manifests when proxy attributes indirectly linked to sensitive attributes influence outcomes. For example, using zip codes in decision-making processes might inadvertently reflect racial biases because residential areas often correlate with racial demographics. This phenomena is also refered as redlining effect~\citep{Pedreschi2008};
    
    \item[Systemic Discrimination] involves policies or practices entrenched within an organization that perpetuate disadvantage for certain groups. This can stem from cultural biases embedded in the decision-making processes, often reflecting the preferences or biases of dominant groups;
    
    \item[Statistical Discrimination] refers to the use of general statistics on a group to make inferences about individuals from that group. This type of discrimination might arise when decision-makers use visible characteristics as proxies for other traits, leading to biased assessments;
    
    \item[Explainable Discrimination] is considered legally permissible if the differences in treatment or outcomes can be justified through legitimate and relevant attributes. For instance, differences in pay might be justified by the number of hours worked if this factor significantly influences earnings;
    
    \item[Unexplainable Discrimination] occurs when there is no justifiable reason for the disparate treatment or outcomes, making it illegal and ethically unacceptable. This type of discrimination requires interventions to ensure fairness and equality in decision-making processes.
\end{description}


\section{Fairness definitions and metrics}

This section aims to present some widely used definitions and metrics of fairness, as described by \cite{Verma2018} and summarized by \cite{Mehrabi2019} and \cite{caton2023}, providing a comprehensive overview for understanding and navigating the multifaceted dimensions of fairness in ML systems. Initially, we explore general considerations and intuitive aspects of fairness, setting the stage for a deeper understanding. This preliminary discussion lays the groundwork for understaining the nuanced nature of fairness notions within the context of ML. Following this, we will transition into formal definitions, where we will dissect and explain those metrics and concepts.

Even before this discussion, we emphasizes that no single fairness definition universally applies to all scenarios. The choice of a particular fairness definition and metric should be informed by ethical considerations grounded in the social context in which the model would be deployed~\citep{AlerTubella2022}. Selecting a fairness definition is not a purely technical matter, as it inevitably requires ethical and social considerations that should not be neglected~\citep{Alves2023}. Building fair machine learning models requires an interdisciplinary approach that engages all stakeholders, including specially those who are typically marginalized or underrepresented~\citep{Weinberg2022}.

A prevalent taxonomy within fairness literature differentiates fairness notions into group metrics and individual metrics. Group Fairness Metrics hinge on the principle that statistical measures — such as error rates, precision, and recall — ought to be equitably distributed across groups demarcated by sensitive attributes like race, gender, or age. The core premise of these metrics is that fairness is actualized when an algorithm exhibits consistent performance across diverse demographic segments.

Demographic Parity, for example, mandates uniformity in the rate of positive algorithmic outcomes across different groups, a standard that remains agnostic to the underlying base rates within each population segment. On the other hand, Equal Opportunity and Equalized Odds introduce a nuance to this conversation by tethering fairness to the true condition of outcomes. This refinement delineates a central differentiation within fairness metrics: some are predicated solely on predicted values (such as Demographic Parity), while others derive from the scope of the confusion matrix (Table~\ref{tab:confusion_matrix_definition}), also incorporating true value conditions (as seen in Equal Opportunity and Equalized Odds).

Individual Fairness Metrics, in contrast, introduce a more granular perspective to fairness, advocating that similar individuals should be treated similarly by the ML system. This approach diverges from group-level considerations, focusing instead on ensuring that the algorithm’s treatment is consistent for individuals who are alike in relevant aspects, barring their membership in different demographic categories. Individual fairness seeks to ensure a personalized sense of justice, where the algorithmic outcomes are solely reflective of pertinent attributes rather than biased by irrelevant factors associated with sensitive attributes. This concept champions the notion that fairness extends beyond group identities to recognize and respect the uniqueness of individual experiences and qualifications.

To establish the foundation for discussing fairness definitions and metrics, we commence with an examination of the confusion matrix, which is an essential instrument in machine learning to assessing the performance of classification algorithms. It constitutes a tabular visualization that delineates the correspondence between the true labels and the predicted outcomes generated by a model. For binary classification tasks, the confusion matrix is structured into four principal components: True Positives (TP), True Negatives (TN), False Positives (FP), and False Negatives (FN), as outlined in Table~\ref{tab:confusion_matrix_definition}. By providing a clear breakdown of these outcomes, the confusion matrix allows to calculate many key performance metrics such as accuracy, precision, recall, and the F1 score, providing comprehensive insights into the strengths and weaknesses of the classification model. Also, the computation of those metrics forms the basis for evaluating fairness across distinct demographic groups.

\begin{table}[h]
    \centering
    \caption{Confusion matrix of binary classification outcomes} \label{tab:confusion_matrix_definition}
    \begin{tabular}{cc|c|c|}
    \cline{3-4}
     & & \multicolumn{2}{c|}{\textbf{Predicted}} \\ \cline{3-4}
     & & Positive & Negative \\ \hline
    \multicolumn{1}{|c|}{\multirow{2}{*}{\textbf{Actual}}} & Positive & TP & FN \\ \cline{2-4}
    \multicolumn{1}{|c|}{} & Negative & FP & TN \\ \hline
    \end{tabular}
\end{table}


True Positives (TP) can be defined as the probability that the predictor correctly identifies a positive outcome when the true condition is positive. Using the conditional probability notation, it is expressed as $P(\hat{Y}=1|Y=1)$, indicating the probability that the predicted class $\hat{Y}$ is positive given that the actual class $Y$ is positive.

False Positives (FP) represent the probability that the predictor incorrectly identifies a positive outcome when the true class is negative. It is denoted as $P(\hat{Y}=1|Y=0)$, reflecting the probability that the predicted class $\hat{Y}$ is positive when the actual class $Y$ is negative.

False Negatives (FN) are defined as the probability that the predictor incorrectly identifies a negative outcome when the true class is positive. This is given by $P(\hat{Y}=0|Y=1)$, the probability that the predicted class $\hat{Y}$ is negative given that the actual class $Y$ is positive.

True Negatives (TN) correspond to the probability that the predictor correctly identifies a negative outcome when the true condition is negative. In conditional probability terms, it is $P(\hat{Y}=0|Y=0)$, indicating the probability that the predicted class $\hat{Y}$ is negative given that the actual class $Y$ is negative.

Now we proceed to more complex metrics that provides complementary insights into the performance of the classifier. These derived metrics, such as Positive Predictive Value (PPV), False Discovery Rate (FDR), and others, constitutes the basic elements of the confusion matrix to quantify the reliability of the predictions in various ways. By expressing these metrics in terms of conditional probabilities and confusion matrix components, we facilitate a comprehensive analysis of the classifier's behavior, providing resources to a proper evaluation of its fairness across different demographic groups.

\begin{definition}[Positive Predictive Value (PPV)]\label{def:ppv}
PPV, or precision, measures the proportion of correctly identified positive outcomes among all predicted positives. It is defined as the probability that the true condition is positive given the predicted condition is positive, $P(Y=1|\hat{Y}=1)$. In terms of the confusion matrix, PPV is calculated as $\frac{TP}{TP + FP}$, the ratio of true positives to the sum of true positives and false positives.
\end{definition}

\begin{definition}[False Discovery Rate (FDR)]\label{def:fdr}
FDR quantifies the rate of incorrect positive predictions. It is the probability that the true condition is negative when the predicted condition is positive, $P(Y=0|\hat{Y}=1)$. From the confusion matrix, FDR is computed as $\frac{FP}{TP + FP}$, indicating the proportion of false positives out of all predicted positives.
\end{definition}

\begin{definition}[Negative Predictive Value (NPV)]\label{def:npv}
NPV assesses the accuracy of negative predictions, representing the probability that the true condition is negative given the predicted condition is negative, $P(Y=0|\hat{Y}=0)$. NPV is derived from the confusion matrix as $\frac{TN}{TN + FN}$, the number of true negatives over the sum of true negatives and false negatives.
\end{definition}

\begin{definition}[False Omission Rate (FOR)]\label{def:for}
FOR indicates the likelihood of a false negative prediction. It corresponds to the probability that the true condition is positive when the predicted condition is negative, $P(Y=1|\hat{Y}=0)$. In the confusion matrix context, FOR is $\frac{FN}{TN + FN}$, representing the number of false negatives relative to all predicted negatives.
\end{definition}

\begin{definition}[True Positive Rate (TPR)]\label{def:tpr}
TPR, or recall, measures the proportion of actual positives that are correctly predicted. It is the probability that the predicted condition is positive given the true condition is positive, $P(\hat{Y}=1|Y=1)$. TPR is calculated as $\frac{TP}{TP + FN}$ in the confusion matrix, the ratio of true positives to the sum of true positives and false negatives.
\end{definition}

\begin{definition}[False Negative Rate (FNR)]\label{def:fnr}
FNR quantifies the rate of missed positive predictions. It is defined as the probability that the predicted condition is negative when the true condition is positive, $P(\hat{Y}=0|Y=1)$. FNR is derived from the confusion matrix as $\frac{FN}{TP + FN}$, indicating the proportion of false negatives out of the actual positives.
\end{definition}

\begin{definition}[True Negative Rate (TNR)]\label{def:tnr}
TNR, or specificity, indicates the accuracy of negative predictions, representing the probability that the predicted condition is negative given the true condition is negative, $P(\hat{Y}=0|Y=0)$. From the confusion matrix, TNR is computed as $\frac{TN}{TN + FP}$, the number of true negatives to the sum of true negatives and false positives.
\end{definition}

\begin{definition}[False Positive Rate (FPR)]\label{def:fpr}
FPR assesses the likelihood of incorrect negative predictions, calculated as the probability that the predicted condition is positive when the true condition is negative, $P(\hat{Y}=1|Y=0)$. FPR is given by $\frac{FP}{TN + FP}$ in the confusion matrix, the ratio of false positives to the sum of true negatives and false positives.
\end{definition}

As we transition from foundational metrics that directly stem from the confusion matrix, such as PPV and TPR, we now discuss standard performance metrics that assess classification models in a more comprehensive manner. These metrics, such as Accuracy and F1 are distinguished by their reliance on both classes to provide a more holistic evaluation.

\begin{definition}[Accuracy (Acc.)]\label{def:acc}
Probably the most widely usaed performence metric to classification problems, Accuracy is the proportion of true results, both true positives and true negatives, among the total number of cases examined. In terms of conditional probabilities, accuracy reflects the probability that the predicted condition is correct, both as a positive and negative outcome, given the actual conditions, and can be 
expressed as 
\begin{align*}
    P(\hat{Y}=Y) &= P(\hat{Y}=1|Y=1) \cdot P(Y=1) \\
    &+ P(\hat{Y}=0|Y=0) \cdot P(Y=0).
\end{align*}
Using the confusion matrix, accuracy is computed as $$\frac{TP + TN}{TP + TN + FP + FN}.$$
\end{definition}
    
\begin{definition}[Balanced Accuracy (Bal. Acc.)]\label{def:ba}
Balanced accuracy is an average of the true positive rate (TPR) and the true negative rate (TNR), which compensates for class imbalance by treating both classes equally. Using conditional probabilities, it can be expressed as $$\frac{1}{2}\left[ P(\hat{Y}=1|Y=1) \; + \; P(\hat{Y}=0|Y=0)\right],$$ where each term represents the conditional probability of correctly predicting the respective class. In terms of the confusion matrix, balanced accuracy is calculated as $$\frac{1}{2}\left[\frac{TP}{TP + FN} + \frac{TN}{TN + FP}\right].$$
\end{definition}
    
\begin{definition}[F1 Score]\label{def:f1}
The F1 score is the harmonic mean of precision and recall, providing a balance between the PPV and TPR. It is calculated as $2 \cdot \frac{\text{PPV} \cdot \text{TPR}}{\text{PPV} + \text{TPR}}$. Using conditional probabilities and confusion matrix terms, the F1 score can be expressed as $$2 \cdot \frac{P(Y=1|\hat{Y}=1) \cdot P(\hat{Y}=1|Y=1)}{P(Y=1|\hat{Y}=1) + P(\hat{Y}=1|Y=1)},$$ and calculated using terms from confusion matrix as $$\frac{2 \cdot TP}{2 \cdot TP + FP + FN}.$$
\end{definition}

\begin{definition}[Matthews Correlation Coefficient (MCC)]\label{def:mcc}
MCC is a measure of the quality of binary classifications, producing a value between -1 and 1 where 1 is a perfect prediction, 0 no better than random prediction, and -1 indicates total disagreement between prediction and observation. The MCC is defined as $$\frac{TP \cdot TN - FP \cdot FN}{\sqrt{(TP+FP) \cdot (TP+FN) \cdot (TN+FP) \cdot (TN+FN)}}.$$ In terms of conditional probabilities, MCC considers all four quadrants of the confusion matrix, correlating the true and predicted conditions. It can be seen as a correlation coefficient between the observed and predicted binary classifications, providing a more informative measure than simple accuracy in the presence of class imbalance.
\end{definition}
    
Now we describe the most widely used group fairness definitions, including statistical parity, equal opportunity, predictive equality, and equalized odds. Demographic parity requires that the likelihood of a positive outcome is the same across different groups, irrespective of their sensitive attributes. Equal opportunity extends this concept to the true positive rate, ensuring that individuals from different groups have an equal chance of being correctly classified as positive. Predictive equality, on the other hand, focuses on the true negative rate, ensuring that individuals from different groups have an equal chance of being correctly classified as negative. Equalized odds combines the principles of equal opportunity and predictive equality, ensuring that both true positive and true negative rates are equal across different groups.

\begin{definition}[Statistical Parity]\label{def:demo_parity}
The likelihood of a positive, i.e., favorable, outcome should be the same in every group of the sensitive attribute~\citep{Dwork2011,Kusner2018}. A binary predictor $\hat{Y}$ satisfies Statistical Parity (a.k.a. Demographic Parity) if $P(\hat{Y}|A=0) = P(\hat{Y}|A=1)$, where $A$ is a protected attribute.
\end{definition}

For example, the credit approval probability should be the same for the male and female groups. Demographic Parity does not depend on true class $Y$, only on prediction $\hat{Y}$. We can measure Demographic Parity (Definition~\ref{def:demo_parity}) for a protected attribute $A$ as the absolute difference between $P(\hat{Y}|A=0)$ and $P(\hat{Y}|A=1)$, as seen in Equation~\ref{eq:demo_parity}. According Demographic Parity, the predictor is considered fairer when this metric is lower.

\begin{equation}\label{eq:demo_parity}
    |P(\hat{Y}|A=0)-P(\hat{Y}|A=1)|
\end{equation}
By analyzing the confusion matrix, we can determine the absolute difference between the rates of $(TP + FP)/(TP + FP + TN + FN)$ for both protected and unprotected groups.

\begin{definition}[Equal Opportunity]\label{def:eq_opp}
The probability of a person in a positive class being assigned to a positive, i.e., favorable, outcome should be the same in every group of the sensitive attribute~\citep{Hardt2016}. A binary predictor $\hat{Y}$ satisfies Equal Opportunity if $P(\hat{Y}|A=0,Y=1) = P(\hat{Y}|A=1,Y=1)$, where $Y$ is true class and $A$ is a protected attribute.
\end{definition}

Definition~\ref{def:eq_opp} claims that protected and unprotected, i.e., privileged, groups should have equal true positive rates. Mathematically, a classifier with equal true positive rates will also have equal false negative rates, so we can analyze the confusion matrix checking whether a predictor has equal $(TP)/(TP + FN)$ or $(FN)/(TP + FN)$ in each group of the sensitive attribute. Like in Demographic Parity, we can measure Equal Opportunity as an absolute difference between protected and privileged groups, as defined in Equation~\ref{eq:eq_opp}.
\begin{equation}\label{eq:eq_opp}
    |P(\hat{Y}|A=0,Y=1)\; - \; P(\hat{Y}|A=1,Y=1)|
\end{equation}

\begin{definition}[Predictive Equality]\label{def:pred_eq}
The probability of a person in a negative class being assigned to a negative outcome should be the same in every group of the sensitive attribute. A binary predictor $\hat{Y}$ satisfies Predictive Equality if $P(\hat{Y}|A=0,Y=0) = P(\hat{Y}|A=1,Y=0)$, where $Y$ is true class and $A$ is a protected attribute.
\end{definition}

Definition~\ref{def:pred_eq} establishes that that both the protected and privileged groups should have the same true negative rates, which consequently results in equal false positive rates. Using a confusion matrix definition, we check the absolute difference of $(TN)/(TN + FP)$ or $(FP)/(TN + FP)$ between the groups. So, we can measure Predictive Equality according Equation~\ref{eq:pred_eq}.
\begin{equation}\label{eq:pred_eq}
    |P(\hat{Y}|A=0,Y=0) \; - \; P(\hat{Y}|A=1,Y=0)|
\end{equation}

\begin{definition}[Equalized Odds]\label{def:eq_odds}
Both probabilities of the person in a positive class being assigned to a positive outcome and of a person in a negative class being assigned to a negative outcome should be the same in every group of the sensitive attribute~\citep{Hardt2016}. A binary predictor $\hat{Y}$ satisfies Equalized Odds (a.k.a. Average Odds Difference) if $P(\hat{Y}|A=0,Y) = P(\hat{Y}|A=1,Y)$, where $Y$ is true class and $A$ is a protected attribute.
\end{definition}

Equalized Odds is a combination of the principles from Definition~\ref{def:eq_opp} and Definition~\ref{def:pred_eq}, i.e., protected and unprotected groups should have equal true positive and true negative rates, therefore equal false positive and false negative rates. Using a confusion matrix definition, we check the absolute difference between $(TP)/(TP + FN)$ and $(TN)/(TN + FP)$ of predictor in protected and unprotected groups. Equation~\ref{eq:eq_odds} describes how to measure Equalized Odds as the average between Equal Opportunity and Predictive Equality. According to Definition~\ref{def:eq_odds}, the predictor is considered fairer when this metric is lower.
\begin{equation}\label{eq:eq_odds}
    \frac{1}{2}\left[\;
    \begin{aligned}
    |P(\hat{Y}|A=0,Y=1)\;-\;&P(\hat{Y}|A=1,Y=1)| \, +\\
    |P(\hat{Y}|A=0,Y=0)\;-\;&P(\hat{Y}|A=1,Y=0)|
    \end{aligned}
    \;\right]
\end{equation}

Using the same logic, it is possible to define group fairness metrics based derived from any binary classification metric from confusion matrix. The procedure is the same, asessing the absolute difference from those metrics between protected and unprotected groups.

While individual fairness metrics strive to ensure equitable treatment of individuals based on their specific attributes and circumstances, they may not always capture broader systemic inequalities that affect entire groups. As we pivot our discussion towards group fairness, we examine the potential drawbacks and complications that can arise when pursuing fairness metrics across different demographic groups.

In this context a key challenge is the Simpson's Paradox~\citep{Simpson}, where trends apparent in separate groups disappear or reverse when these groups are combined. This can lead to misleading conclusions in aggregated data, potentially obscuring significant disparities within subgroups that are averaged out in the analysis. Furthermore, group fairness metrics may inadvertently mask discrimination within protected groups. For instance, a model could satisfy group fairness criteria overall while still discriminating against specific subgroups within a protected class due to the heterogeneity within larger groups that isn't captured by broader fairness assessments.

Additionally, implementing group fairness often involves trade-offs~\cite{Goh2016,Komiyama2018,Petrovic2021,Cruz2021,Liu2022} that can impact the overall performance of the predictive model. Balancing fairness with accuracy can lead to difficult choices, especially in high-stakes applications such as healthcare or criminal justice, where the cost of errors is significant. For example, efforts to reduce false positive rates in one group might inadvertently increase false negatives in another, adversely affecting the model's overall predictive utility. Another issue arises from the conflict between different fairness definitions, where improving fairness according to one metric might worsen it according to another. Achieving demographic parity, which calls for equal outcomes across groups, might conflict with ensuring equal opportunity, which demands equal true positive rates across groups. Such conflicts necessitate careful consideration to determine which fairness criteria are most appropriate for specific applications.

Lastly, standard group fairness metrics often overlook intersectionality—the complex, cumulative way in which multiple forms of discrimination, such as race, gender, and class, intersect and affect individuals~\citep{Kearns2017,Kearns2018}. Ignoring this aspect can result in policies and models that do not fully address the nuanced ways in which bias manifests. This oversight underscores the importance of individual fairness, a principle that seeks to ensure equitable treatment by focusing on the uniqueness of each individual rather than merely categorizing them into groups. Individual fairness advocates for algorithms to treat similar individuals similarly, regardless of their group membership~\cite{Mehrabi2019}, thus acknowledging and addressing the multifaceted nature of discrimination and ensuring that each person is considered on their own merits. By integrating individual fairness into our models, it is possible to better capture and mitigate the intersecting and often overlapping biases that group fairness metrics might miss, providing a more comprehensive approach to fairness in AI systems

Fairness Through Awareness~\citep{Dwork2011} is a concept which focuses on treating similar individuals similarly. It emphasizes the importance of fairness at the individual level by defining a metric of similarity between individuals based on relevant characteristics, and ensuring that the algorithm's decisions are consistent for individuals deemed similar by this metric. This approach is rooted in the idea that fairness can be achieved by explicitly considering the sensitive attributes through a carefully defined similarity function, ensuring that decisions are justifiable and tailored to individual circumstances.


\begin{definition}[Fairness Through Awareness]\label{def:fta}
A predictor $\hat{Y}$ satisfies Fairness Through Awareness if for any two individuals $x, x' \in X$, where $X$ is the domain of individuals, the distance metric $d(x, x')$ under which the individuals are considered similar enforces that $|\hat{Y}(x) - \hat{Y}(x')| \leq d(x, x')$. Here, $d$ is a task-specific metric that measures similarity relevant to the decision-making process, incorporating sensitive attributes where necessary.
\end{definition}

This definition implies that the algorithm must incorporate a nuanced understanding of what it means for two individuals to be similar, which goes beyond merely ignoring sensitive attributes. Instead, it considers these attributes in a way that respects individual differences and upholds fairness.

Fairness Through Unawareness~\cite{Corbett-Davies2018}, on the other hand, is a more straightforward approach where an algorithm is considered fair if it does not explicitly use sensitive attributes (such as race, gender, etc.) in the decision-making process. This method assumes that the exclusion of sensitive attributes will prevent discriminatory practices. However, this approach can be naive as it fails to consider that biases can be encoded in other, non-sensitive attributes that are correlated with the sensitive ones~\cite{Mehrabi2019,caton2023,Hort2023}.

\begin{definition}[Fairness Through Unawareness]\label{def:ftu}
A predictor $\hat{Y}$ satisfies Fairness Through Unawareness if the decision function $\hat{Y}$ does not explicitly include any sensitive attribute $A$ as part of the input. In other words, $\hat{Y}$ is constructed without direct knowledge of $A$.
\end{definition}

Another example of Individual Fairness Metric is the notion of Counterfactual Fairness~\citep{Kusner2018}, which introduce a causal reasoning framework into the fairness discourse. These metrics are based on the concept that a decision is fair towards an individual if the same decision would have been made in a counterfactual world where the individual belonged to a different demographic group but all other characteristics remained constant. This approach hinges on causal models that specify how sensitive attributes affect other features and the outcome. Counterfactual fairness aims to address the individual-level biases that group fairness metrics might overlook, providing a nuanced approach that considers the hypothetical scenarios of individuals belonging to different demographic categories. By employing counterfactual analysis, one can assess whether the disparities in ML predictions stem from legitimate factors or unjust biases. Relevant works approaching this notion include~\citet{Wu2022}, \citet{Yuchen2023}, and~\citet{GrariLD23}.

\begin{definition}[Counterfactual Fairness]\label{def:counterfactual_fairness}
A predictor $\hat{Y}$ is counterfactually fair with respect to a protected attribute $A$ if, under any context $X = x$ and $A = a$, the distribution of $\hat{Y}$ is the same in the actual world and a counterfactual world where $A$ is set to any permissible value. That is,
$$
P(\hat{Y}_{A \leftarrow a}(U) = y \mid X = x, A = a) = P(\hat{Y}_{A \leftarrow a'}(U) = y \mid X = x, A = a),
$$
for all $y$ and any value $a'$ of $A$, where $X$ are the features not causally dependent on $A$, and $U$ denotes the background variables.
\end{definition}

This definition roots itself in the idea that fairness should be preserved across hypothetical alterations of the sensitive attribute, reflecting a robust stance against biases that might otherwise emerge due to such attributes.

Implementing counterfactual fairness involves constructing a causal model that maps how inputs (features including sensitive attributes) influence the outputs (predictions). One must identify which attributes are causally independent of the sensitive attribute and ensure that the predictions are invariant when the sensitive attribute's values are modified hypothetically.

This approach is particularly pertinent when decisions have substantial impacts on individuals, such as in hiring, loan approval, or healthcare settings. By ensuring that predictions remain consistent regardless of changes to sensitive attributes, models can be designed to mitigate unfair discriminatory practices that could otherwise affect outcomes based on irrelevant attributes.

While the concept of counterfactual fairness is intuitive and persuasive, its implementation poses significant challenges~\cite{Kasirzadeh2021}. Building models that reflect the true causal relationships in the data is non-trivial and requires deep domain knowledge. Also, access to good quality data that sufficiently captures the causal dependencies is necessary, which can be a limiting factor in many practical scenarios. Finally, the complexity of calculating counterfactuals, especially in large datasets with many attributes, can be computationally demanding. Despite these challenges, counterfactual fairness pushes the boundaries of fairness in machine learning by providing a framework that directly tackles the underlying causal mechanisms leading to biased decisions.

In the context of fairness definitions and metrics there is a relevant problem to be considered, the Impossibility Theorem. As elucidated simultaneously by \cite{Kleinberg2017} and \cite{Chouldechova2017} with further contributions by \cite{Saravanakumar2020}, \cite{Bell2023} and \cite{Beigang2023}, articulates a fundamental challenge in the domain of algorithmic fairness: the concurrent satisfaction of distinct fairness metrics is inherently unfeasible under certain conditions. This theorem , also referred to as the Incompatibility of Fairness Criteria, delineates the intrinsic conflicts arising amongst prevalent fairness constructs.

The Impossibility Theorem in the context of algorithmic fairness articulates a fundamental challenge: it is not feasible to simultaneously satisfy multiple fairness criteria in certain realistic settings. This theorem highlights the inherent conflict that arises when attempting to meet several well-intentioned fairness metrics such as counterfactual fairness, equalized odds, and predictive parity at the same time.

According to the theorem, if a predictive model is designed to achieve counterfactual fairness, it will likely conflict with the criteria of equalized odds or predictive parity. Counterfactual fairness demands that the model's prediction for an individual would remain unchanged in hypothetical scenarios where the individual's protected characteristics (such as race or gender) are altered but all other variables are held constant. In contrast, equalized odds require that error rates across different groups are similar, while predictive parity necessitates comparable predictive values across these groups. When protected characteristics are causally relevant to the predicted outcomes, aligning the model with one fairness metric may inadvertently breach another.

This theorem thus underscores the practical dilemmas in fair machine learning models. Achieving comprehensive fairness in ML systems often requires navigating complex trade-offs, necessitating a thoughtful prioritization of fairness criteria tailored to the specific context and ethical considerations of each use case. The Impossibility Theorem serves as a critical reminder of the limitations and careful considerations required in the pursuit of fair decision making systems, highlighting the importance of making informed, contextually sensitive decisions when implementing fairness metrics.

\section{Fair classification}

In this section, we review pertinent literature on fair machine learning, placing a particular emphasis on in-processing techniques. Fairness intervention methods can be classified into three categories based on the stage at which they occur, as proposed by \cite{Mehrabi2019} and \cite{AlerTubella2022}:

\begin{description}
    \item[Pre-processing] intervene before learning, modifying the data to reduce existing biases;
    \item[In-processing] intervene during learning by modifying the objective functions or imposing constraints to the model in order to mitigate discriminatory effects;
    \item[Post-processing] affects predictions produced by the model after learning to change possibly unfair outcomes. 
\end{description}

One notable pre-processing method is the reweighting approach proposed by \citet{Kamiran2012}, which adjusts the weights of different samples in the training data to ensure that underrepresented groups are fairly represented during training. Another significant pre-processing technique is the Fair Representation Learning by \citet{Zemel2013}, which learns a latent representation of the data that obfuscates sensitive attributes while retaining the information necessary for accurate predictions. An example of a post-processing method is the Reject Option Classification by \citet{Kamiran2012b}, which changes the decisions of the classifier for individuals near the decision boundary. Another example is Equalized Odds and Equal Opportunity post-processing technique by \citet{Hardt2016}, which adjusts the classifier's predictions to equalize the true positive and false positive rates across different demographic groups.

In this work, we incorporate information about disparities among social groups in the dataset into our model by modifying the loss function through the use of a transition matrix. This fairness intervention is thus classified as an in-processing technique. Other relevant in-processing strategies for fair classification include Naive Bayes approaches for discrimination-free classification~\citep{Calders2010}, Fairness Through Awareness Framework~\citep{Dwork2011}, Fairness-Aware Classifier with Prejudice Remover Regularizer~\cite{Kamishima2012}, $\alpha$-discriminatory empirical risk minimizer~\citep{Woodworth2017}, Disparate Impact and Disparate Mistreatment frameworks for margin-based classifiers~\citep{Zafar2017a,Zafar2017b}, Weak Agnostic Learning to Auditing Subgroup Fairness~\citep{kearns18a,Kearns2018}, One-Network Adversarial Fairness~\citep{Adel2019}, FairGan$^{+}$ ~\citep{Xu2019}, Monte Carlo policy gradient method~\citep{Petrovic2021}, Fairness-accuracy Pareto~\citep{Wei2022}, and Pareto front stochastic multi-gradient~\citep{Liu2022} based in original stochastic multi-gradient~\citep{Mercier2018} to Multi-Objective Optimization and the hybrid Adaptive Priority Reweighing approach~\cite{HuXT23}.

In \cite{Kamishima2012} the authors proposes the Prejudice Remover (PR) which is a regularizer to logistic regression models. It introduces an additional term in the loss function to penalize the model for making decisions based on sensitive features. The objective function to be minimized is available on Equation~\ref{eq:prejudice_remover}, where $\Theta$ is the model parameters, $L(D;\Theta)$ the log-likelihood, $R(D, \Theta)$ the prejudice remover regularizer,$\eta$ a regularization parameter controlling the trade-off between fairness and accuracy, and $\lambda$ a parameter for the L2 regularizer.

\begin{equation} \label{eq:prejudice_remover}
    -L(D;\Theta) + \eta R(D, \Theta) + \frac{\lambda}{2} \|\Theta\|^2
\end{equation}

The regularizer \( R(D, \Theta) \) aims to minimize the mutual information between the predicted outcomes and the sensitive features, thereby reducing the model's reliance on sensitive information. The mutual information is approximated using sample means to make the computation feasible for large datasets.The authors compared the proposed method with Calders-Verwer 2-naïve-Bayes method~\citep{Calders2010}, showing that the PR effectively reduced bias, though sometimes at the cost of decreased accuracy. 

As an alternative to mitigating unwanted bias, \cite{Zhang2018} proposes an adversarial method for reducing bias in machine learning models, namely Adversarial Debiasing (AD). This technique involves training a neural network predictor to forecast an outcome variable from inputs while an adversary network simultaneously attempts to predict a sensitive attribute, which should not influence the outcome. 

The method utilizes an adversarial network architecture where the main predictor's task is complemented by an adversarial model that tries to learn the sensitive attribute. By integrating the adversarial model’s feedback into the training process, the predictor learns to make decisions that are increasingly independent of the sensitive attribute. This setup allows the model to attain to fairness constraints like Statistical Parity (Definition~\ref{def:demo_parity}), Equal Opportunity (Definition~\ref{def:eq_opp}), and Equalized Odds (Definition~\ref{def:eq_odds}).

In a similar vein, \cite{kearns18a} propose a framework for ensuring subgroup fairness, addressing the issue of fairness gerrymandering. Below, a toy example given by the authors ilustrating a scenario where the refered fairness gerrymandering occurs.

\begin{quote}\textit{
Imagine a setting with two binary features, corresponding to race (say black and white) and gender (say male and female), both ofwhich are distributed independently and uniformly at random in a population. Consider a classifier that labels an example positive if and only if it corresponds to a black man, or a white woman. Then the classifier will appear to be equitable when one considers either protected attribute alone, in the sense that it labels both men and women as positive $50\%$ of the time, and labels both black and white individuals as positive $50\%$ of the time. But ifone looks at any conjunction of the two attributes (such as black women), then it is apparent that the classifier maximally violates the statistical parity fairness constraint. Similarly, if examples have a binary label that is also distributed uniformly at random, and independently from the features, the classifier will satisfy equal opportunity fairness with respect to either protected attribute alone, even though it maximally violates it with respect to conjunctions oftwo attributes.}
\end{quote}


Their approach involves defining fairness for exponentially or infinitely many subgroups defined by a structured class of functions over the protected attributes, not only for a small number of pre-defined groups as considered in hegemonic fair classification approaches. This framework is formalized as a two-player zero-sum game between a Learner (the primal player) and an Auditor (the dual player), where the Learner aims to minimize classification error while the Auditor seeks to identify and penalize fairness violations. The computational challenges of this approach are mitigated by connecting it to weak agnostic learning, which allows the use of practical machine learning heuristics for effective auditing and learning in real-world applications.

The algorithms derived from this framework provably converge to the best fair distribution over classifiers, given access to oracles capable of optimally solving the agnostic learning problem. These algorithms include a variant based on the no-regret Follow the Perturbed Leader algorithm and another using Fictitious Play, both of which have been implemented and evaluated on real datasets, demonstrating their efficacy in achieving subgroup fairness.


The Adaptive Priority Reweighing~\cite{HuXT23} method introduces a systematic approach to increase fairness and generalizability of classifiers by dynamically adjusting sample weights based on their proximity to the decision boundary. Initially, the training samples are divided into subgroups according to their sensitive attributes and classifier predictions. Each sample's distance to the decision boundary is then computed and updated iteratively. During each iteration, subgroup weights are recalibrated by comparing the observed probability of the positive prediction rate within each subgroup to the expected probability under statistical independence. This comparison helps to assign higher weights to samples closer to the decision boundary, thereby prioritizing them in the training process. The weighted loss function is optimized using a stochastic gradient descent algorithm, which iteratively adjusts the classifier to reduce bias while maintaining accuracy. By continuously updating the weights and distances, the Adaptive Priority Reweighing method ensures that the classifier learns to make fairer decisions that generalize well to unseen data, addressing the limitations of traditional reweighing methods that often fail to generalize beyond the training set.

The method is evaluated on benchmark datasets, outperforming many state-of-art pre-processing, in-processing and post-processing fair classification techniques, promoting fairness on both finetuned pre-trained models and newly traisned models. This technique can be classified as an hybrid approach, performing the traditional pre-processing instance reweighing through an adaptive training algorithm.

Recently, special attention has been given in fair machine learning research topics like addressing multiple sensitive attributes or multiple classes \cite{DAloisio2023,Liu_Wang_Wang_Wang_Su_Gao_2023}, loss balancing techniques~\cite{KIM2023231,KhaliliZA23}, where the objective is to balance the loss across different groups instead of predictive metrics, adversarial approaches~\cite{Liang2023,ZhangZLZY23,GrariLD23,MousaviMD23,Zeming2023,Yuchen2023} and the privacy concerns involving fairness under federated learning settings~\cite{ChenZZZY24,VucinichZ23}. Another relevant research topic in fair machine learning is learning under censored data~\cite{WZhang2022,WZhang2023_a,WZhang2023_b,WZhang2023_c}, which we will discuss in section~\ref{sec:fairness_noise}.
  \include{03_fair_transition_loss}
  \chapter{Correlation based penalty function}

In this chapter, we propose a novel regularization factor to penalize the use of features highly correlated with a protected attribute by a machine learning model, aiming to avoid the redlining effect~\citep{Pedreschi2008}. The proposed penalization factor is applied directly to the input weights of a Multi-Layer Perceptron, using a strategy that proportionately penalizes features use based on their correlation with the sensitive feature.

\section{Preliminaries}

The phenomena known as redlining effect~\citep{Pedreschi2008} consists in the unintended use of proxy variables to the sensitive feature by the model, which can lead model to produce indirect discrimination in their outcomes.  Thus, the insight here is to penalize the use of those proxy features by the model proportionally according its correlation with sensitive feature in order to avoid redlining effed. The proposed regularization approach uses the recently described Chatterjee's xi correlation coefficient~\cite{chatterjee2020new}, which robustly asses whether a random variable can be described as a function of another one, as a measure of the potential of given feature to be used by the model as a proxy to the sensitive feature. 

Before discussing this approach, we start defining some common correlation coefficients and providing a proper comparison within Chatterjee's correlation. For purpose of simplicity we describe only the most commonly used form, more complex formulations involving additional terms depending on available data should be considered to most of coefficients. Also, we present some related approaches, specially those ones that, like ours, use regularization and penalty factors to avoid indirect discrimination.

The Pearson correlation coefficient, denoted by $\rho$ and defined in Definition~\ref{def:pearson}, is a measure of the linear relationship between two random variables. The Pearson correlation coefficient ranges from $-1$ to $1$, where $1$ indicates a perfect positive linear relationship, $-1$ indicates a perfect negative linear relationship, and $0$ indicates no linear relationship.

\begin{definition}[Pearson correlation coefficient]\label{def:pearson}
Let  $X$ and $Y$\ random variables. The Pearson correlation coefficient $\rho$ between $X$ and $Y$ can be defined as  
\begin{equation}
\rho = \frac{\mathrm{Cov}(X, Y)}{\sigma_X \sigma_Y},
\end{equation}
where $\mathrm{Cov}(X, Y)$ is the covariance between $X$ and $Y$, while $\sigma_X$ and $\sigma_Y$ the standard deviations of $X$ and $Y$, respectively. 
\end{definition}

Spearman's rank correlation coefficient, denoted by $\rho_s$ and defined in Definition~\ref{def:spearman}, measures the strength and direction of the monotonic relationship between two ranked (ordered) variables.  As like Pearson correlation coefficient, Spearman's rank correlation coefficient ranges from $-1$ to $1$, where $1$ indicates a perfect positive relationship, $-1$ indicates a perfect negative relationship, and $0$ indicates no relationship.
 
\begin{definition}[Spearman's correlation coefficient]\label{def:spearman}
Let $X$ and $Y$\ random variables, $n$ the sample size and $X_i, \, Y_i$ the $i$-th observations to $i = 1 \ldots n$. Let $R_{X_i}, \, R_{Y_i}$ the rank of observations $X_i, \, Y_i$, i.e. the position of $X_i, \, Y_i$ by ordering the samples, respectively. The Spearman's rank correlation coefficient  $\rho_s$ between $X$ and $Y$ can be defined to non-repeated observation values as 
\begin{equation}\label{eq:spearman}
\rho_s = 1 - \frac{6 \sum\limits_{i=1}^{n}d_i^2}{n(n^2 - 1)},
\end{equation}
where $d_i = R_{X_i} - R_{Y_i}$.
\end{definition}

Kendall's rank correlation coefficient, denoted by $\tau$ and defined in Definition~\ref{def:kendall},  indicates the strength and direction of association between two variables. It is based on the relative ordering of pairs of observations rather than their actual values. Here we interpret the correlation values as like in Spearman's and Pearson's correlation coefficients, ranging from $-1$ to $1$, where $1$ indicates a perfect positive relationship, $-1$ a perfect negative relationship, and $0$  no linear relationship.

\begin{definition}[Kendall's correlation coefficient]\label{def:kendall}
Let two variables $X$ and $Y$ sampled with $n$ pairs of observations $(X_i, Y_i)$ for $i = 1, 2, \ldots, n$. A pair of observations $(X_i, Y_i)$ and $(X_j, Y_j)$ is \textbf{concordant} if the ranks (order) of both elements agree, i.e., to $i < j$ either $X_i > X_j$ and $Y_i > Y_j)$ or $X_i < X_j$ and $Y_i < Y_j$. Otherwise their are considered \textbf{discordant}, i.e., either $X_i > X_j$ and $Y_i < Y_j$ or $X_i < X_j$ and $Y_i > Y_j$. The Kendall's rank correlation coefficient $\tau$ between $X$ and $Y$ can be defined as 
\begin{equation}\label{eq:kendall}
\tau = \frac{2(N_c - N_d)}{n(n-1)},
\end{equation}
where $N_c$ is the number of concordant observation and $N_d$ the number of discordant observations.
\end{definition}

Chatterjee's rank correlation coefficient, denoted by $\xi$ and defined in Definition~\ref{def:chatterjee}, is designed to robustly measure the degree of dependence between two variables without assuming any specific type of relationship and capturing noise nuances.  The previously described correlation coefficients are not effective on detecting associations that are not monotonic, even in complete absence of noise.

This correlation coefficient asses whether a random variable can be described as a function of another one. Differently from correlation coefficients described before, this coefficient ranges from 0 to 1, where 0 indicates independence and 1 indicates a perfect functional relationship. Also this correlation is not symmetric, i.e., the correlation between $X$ and $Y$ may differ from between $Y$ and $X$. 

\begin{definition}[Chatterjee's correlation coefficient]\label{def:chatterjee}
Let two variables $X$ and $Y$ , where $Y$ is not a constant, sampled with $n$ pairs of observations $(X_i, Y_i)$ for $i = 1, 2, \ldots, n$. The Chatterjee's rank correlation coefficient $\tau$ between $X$ and $Y$ when there are no ties among $Y$ can be defined as 
\begin{equation}\label{eq:chatterjee}
\xi = 1 - \frac{3 \sum\limits_{i=1}^{n-1} |R_{i+1} - R_i|}{n^2 - 1},
\end{equation}
where $R_i$ is the rank of $Y_i$ in the ordered sequence of $Y$ values corresponding to the sorted $X$ values. 
\end{definition}

To illustrate some characteristics of Chatterjee's rank correlation coefficient we compare those results on Anscombe's quartet~\citep{anscombe1973} along with Pearson's, Spearman's and Kendall's correlation coefficients. The Anscombe's quartet (Table~\ref{tab:anscombe}) is a set of four different datasets that have nearly identical simple descriptive statistics, yet very different distribution, which is clear on Figure~\ref{fig:anscombe}. As an additional resource, a line representing a linear regression over the data is plotted within the points, enforcing that they present nearly identical simple descriptive statistics.

\begin{table}[ht]
\centering
\caption{Anscombe's quartet}
\label{tab:anscombe}
\begin{tabular}{rr|rr|rr|rr}
\toprule
\multicolumn{2}{c|}{I} & \multicolumn{2}{c|}{II} & \multicolumn{2}{c|}{III} & \multicolumn{2}{c}{IV} \\

\multicolumn{1}{c}{$x$} & \multicolumn{1}{c|}{$y$} & \multicolumn{1}{c}{$x$} & \multicolumn{1}{c|}{$y$} & \multicolumn{1}{c}{$x$} & \multicolumn{1}{c|}{$y$} & \multicolumn{1}{c}{$x$} & \multicolumn{1}{c}{$y$} \\
\midrule
10.0 & 8.04 & 10.0 & 9.14 & 10.0 & 7.46 & 8.0 & 6.58 \\
8.0 & 6.95 & 8.0 & 8.14 & 8.0 & 6.77 & 8.0 & 5.76 \\
13.0 & 7.58 & 13.0 & 8.74 & 13.0 & 12.74 & 8.0 & 7.71 \\
9.0 & 8.81 & 9.0 & 8.77 & 9.0 & 7.11 & 8.0 & 8.84 \\
11.0 & 8.33 & 11.0 & 9.26 & 11.0 & 7.81 & 8.0 & 8.47 \\
14.0 & 9.96 & 14.0 & 8.10 & 14.0 & 8.84 & 8.0 & 7.04 \\
6.0 & 7.24 & 6.0 & 6.13 & 6.0 & 6.08 & 8.0 & 5.25 \\
4.0 & 4.26 & 4.0 & 3.10 & 4.0 & 5.39 & 19.0 & 12.50 \\
12.0 & 10.84 & 12.0 & 9.13 & 12.0 & 8.15 & 8.0 & 5.56 \\
7.0 & 4.82 & 7.0 & 7.26 & 7.0 & 6.42 & 8.0 & 7.91 \\
5.0 & 5.68 & 5.0 & 4.74 & 5.0 & 5.73 & 8.0 & 6.89 \\
\bottomrule
\end{tabular}
\end{table}

Comparing correlation coefficients, it is evident that Pearson's $\rho$ does not properly capture the peculiarities of Anscombe's quartet. Although the four datasets present the same correlation coefficient, they exhibit very different distribution. Spearman's $\rho_s$ and Kendall's $\tau$ performs very similarly each other, properly capturing relevant characteristics, albeit Kendall's $\tau$ demonstrates to be more exigent on assigning high correlation values. Chatterjee's $\xi$ is even more exigent, pursuing the behavior of capturing functional relations between data, considering noise. For example, the first quartet presents a small $\xi$ as it contains relevant noise, despite the points are effectively linearly disposed. Albeit the second quartet demonstrate a non linear distribution, the data has lower noise influence, presenting a more functional relationship. Thus, to Chatterjee's correlation the second presents a higher coefficient than the first, which is not equivalently captured by Spearman's and Kendall's. 
\begin{figure}[!ht]
\centering
\caption{Multiple correlation coefficients on Anscombe's quartet.}\label{fig:anscombe}
    \includegraphics[width=1\linewidth]{images/anscombe_quartet.pdf}
\end{figure}



As a resource to defining the proposed redlining penalpy method we describe the L2 regularization. This regularization term, a.k.a. weight decay, is a common technique used to prevent overfitting in machine learning models, including Multi-Layer Perceptrons (MLPs). The L2 regularization adds a penalty term to the loss function that is proportional to the sum of the squares of the model parameters (weights). This encourages the model to keep the weights small, which can help improve generalization.

Let $\mathbf{W}^{(l)}$ represent the weight matrix for the $l$-th layer of the MLP, and let $\mathbf{b}^{(l)}$ denote the corresponding bias vector. The primary loss function of the network, $L_0$, could be any suitable loss function such as the mean squared error for regression or the cross-entropy loss for classification. The L2 regularization term for a single layer is given by
\begin{equation}
R(\mathbf{W}^{(l)}) = \frac{1}{2} \sum_{i=1}^{d_l} \sum_{j=1}^{h_l} \left( W^{(l)}_{ij} \right)^2,
\end{equation}
where $d_l$ and $h_l$ are the dimensions of the weight matrix $\mathbf{W}^{(l)}$, and $W^{(l)}_{ij}$ is the weight connecting the $i$-th input neuron to the $j$-th neuron in the $l$-th layer. Thus, the total regularization term for the entire network, considering all layers, is
\begin{equation}
R(\mathbf{W}) = \frac{1}{2} \sum_{l=1}^{L} \sum_{i=1}^{d_l} \sum_{j=1}^{h_l} \left( W^{(l)}_{ij} \right)^2,
\end{equation}
where $L$ is the total number of layers in the network. Furthermore, the total loss function $L$ for the MLP, incorporating the L2 regularization term, is defined as
\begin{equation}
L = L_0 + \lambda \; R(\mathbf{W}),
\end{equation}
where $\lambda$ is a scalar hyperparameter that controls the overall strength of the regularization.

By adding this regularization term, the optimization process aims to minimize the primary loss $L_0$ along with keeping the weights small, thereby helping to reduce the model complexity and prevent overfitting. The gradient descent updates for the weights will be adjusted to account for the regularization term, effectively shrinking the weights during the training process.

~~\cite{Kamishima2012}

\section{Redlining Penalty Regularizer}

Here we propose the Redlining Penalty Regularizer (RPR), a novel regularization term that penalizes the weights of features highly correlated with the sensitive attribute in order to prevent the redlining effect. By incorporating this penalty into the loss function of the neural network, the model is encouraged to reduce its reliance on sensitive attributes, thus promoting fairer predictions.

As referred before, the Chatterjee Xi Correlation Coefficient distinguish from many other by providing a measure of how much one random variable can be expressed as a function of another, with a range from $0$ to $1$. This characteristics is relevant to the proposed use, to capture redlining effect, as of this phenomena happens exactly when the sensitive feature can be inferred by another one, producing the same harmful effects whether the correlation is positive or negative. Naturally it would be possible to use the absolute value of correlation coefficient such as Pearson, Spearman and Kendall, but this strategy could lead to some kind of information loss. Another relevant characteristics of this correlation coefficient is the ability to capture sophisticated nonlinearities, including nonmonotonic ones, and the effects of noise on variable's distribution, conditions frequently present in proxy feature relationships. 

Thus, let $X \in \mathbb{R}^{n \times d}$ be a dataset where $n$ represents the number of instances and $d$ represents the number of features. Let $X_i \in \mathbb{R}^n$ denote the $i$-th feature of the dataset, and let $A = X_i \in \mathbb{R}^n$ be a sensitive (protected) feature for some $i$. In this neural network, $\mathbf{W}^{(1)} \in \mathbb{R}^{d \times h}$ is the weight matrix for the first hidden layer, with $h$ being the number of neurons in this layer. Additionally, $\lambda$ is a scalar that controls the overall strength of the regularization.

Thus, the proposed regularization term $R(\mathbf{W}^{(1)})$ applied to the weight matrix $\mathbf{W}^{(1)}$ only on the first hidden layer, defined as 
\begin{equation}\label{eq:xi_reg}
R(\mathbf{W}^{(1)}) = \sum_{i=1}^d \xi_n(X_i,\,A) \sum_{j=1}^h (W^{(1)}_{ij})^2,
\end{equation}
where $W^{(1)}_{ij}$ are the weights connecting the $i$-th input feature to the $j$-th neuron in the first hidden layer. Here, the Chatterjee's Xi Correlation Coefficient $\xi_n(X_i,\,A)$ between the $i$-th input feature $X_i$ and the sensitive feature $A$ acts as the regularization strength for the $i$-th input feature. The greater $i$-th input feature dependence on sensitive feature the greater the penalization factor enforcing lower values to those weights.

Thus, the total loss function $L$ to a MLP, incorporating the sensitive-feature-specific $L_2$ regularization, is defined as
\begin{equation}\label{eq:total_regularized_loss}
L = L_0 + \lambda \; R(\mathbf{W}^{(1)}),
\end{equation}
where $L_0$ is the primary loss function of the network. This formulation ensures that the model's learning process penalizes weights with high values associated with features highly correlated to the sensitive attribute, thereby reducing the potential for biased decisions influenced by redlining effect.

This formulation differs from similar approaches to regularization in order to achieve fairness in machine learning by two fundamental points.  The first one is that here the regularization acts as a focused penalty factor through the use of the correlation of each feature to the sensitive one, reducing it's effect on less correlated features and boosting on the highly correlated ones. The second point is that this proposed penalty does not acts as a common regularizer, by imposing small values on every model's parameter. Rather, it penalizes the use of each feature according it's correlation to the sensitive one by exclusively acting on the input layer weights, that is, the usage level of each feature by hidden units.

With the combination of characteristics tailored by this formulation and the chosen correlation coefficient we do expects effectively avoiding the use of indirect sensitive feature predictors within the model, thus mitigating redlining effect.

\section{Experimental setup}

In this section, we detail the experimental setup employed to benchmark the proposed regularization within both Standard MLP with Cross Entropy Loss and a MLP using Fair Transition Loss. Both MLP model uses two hidden layers with $100$ hidden units each, $ReLU$ activation function, batch size of $64$, $50$ epochs early stopped at $3$ epochs without improvement~\citep{Li2020} and softmax in output, trained with ADAM optimizer~\citep{KingmaB14} and learning rate at $3\mathrm{e}{-4}$. 

The overall experimental methodology follows same principles of those used on Fair Transition Loss evaluation. There are two phases: hyperparameter tuning and testing. In the hyperparameter tuning phase we perform a Bandit-Based pruning approach using HyperBand~\citep{Li2018} with Tree-structured Parzen Estimator Sampler (TPE)~\citep{bergstra2011} over $100$ trials. At each trial a fitness function is evaluated by performing a complete training and validation, where both model performance and fairness metrics are assessed. The fitness function is computed based on the objective defined in Equation~\ref{eq:obj_fn}. 

Once the best hyperparameters are selected, we proceed to the testing phase, where a new training is conducted using those optimal hyperparameters. After this training, we evaluate the model's performance on a separate test set that was not used during the hyperparameter tuning phase, which are reported. This complete tuning-training-testing described is repeated $15$ times within random re-sampling then we proceed to comparison. Here the re-sampling consists in shuffling the whole dataset before splitting, as described before

As the objective defined in Equation~\ref{eq:obj_fn} can be achieved with different performance and fairness metrics, we compare the proposed regularization schema within Standard MLP and MLP with FTL in different optimization scenarios, considering as performance metrics Accuracy (Acc.) and Mathews Correlation Coefficient (MCC), while the fairness metrics considered are Statistical Parity (Stat. Parity), Equal Opportunity (Eq. Opp.) and Equalized Odds (Eq. Odds). Those pursued metrics lead us to the following optimization scenarios: MCC and Statistical Parity; MCC and Equal Opportunity; MCC and Equalized Odds; Accuracy and Statistical Parity; Accuracy and Equal Opportunity; Accuracy and Equalized Odds.

\begin{table}[ht]
\centering
\caption{Hyperparameters search ranges or options of each method.}\label{tab:hyperparameters_rpr}
{\footnotesize
\begin{tabular}{lll}
\toprule
Method & Parameter & Range/options \\ \midrule
 Standard MLP without regularization (baseline) & dropout & $[0.0,\,0.2]$  \vspace{1ex} \\
 Standard MLP with RPR & dropout & $[0.0,\,0.2]$  \vspace{1ex} \\
 & $\lambda$ & $\{1e^{-2}, 1^{e-3}, 1^{e-4}\}$ \\
 Fair Transition Loss without regularization & $d_0,p_0,d_1,p_1$ & $[0.0,\,1.0]$ \\
 &  dropout & $[0.0,\,0.2]$ \\
 Fair Transition Loss with RPR & $d_0,p_0,d_1,p_1$ & $[0.0,\,1.0]$ \\
 &  dropout & $[0.0,\,0.2]$ \\
 & $\lambda$ & $\{1e^{-2}, 1^{e-3}, 1^{e-4}\}$ \\
\bottomrule
\end{tabular}
}
\end{table}

Table~\ref{tab:hyperparameters_rpr} presents the methods hyperparameters along with their corresponding search ranges or options. While each method may possess a varying number of hyperparameters and range sizes, all are optimized under the same conditions and number of configurations to guarantee a balanced comparison.

To properly compare this set of $15$ results of each method, we conduct an Almost Stochastic Order (ASO) test \citep{dror2019deep}, a significance test suitable for comparing complex machine learning models with various hyperparameters. As described before, the ASO test involves evaluating a set of metrics through multiple samplings of a Collection of Statistics (in this case assessed in test phase using random resampling) to compare one method against another. The $ASO(A, B)$ function yields a value in the range $[0, 1]$, given two methods $A$ and $B$. If $ASO(A, B)$ is lesser than $0.5$, we can reject the null hypothesis and conclude that method $A$ outperforms method $B$ in the given task. That is, method $A$ produces stochastically larger values than method $B$ for a given metric. The lower the $ASO(A, B)$ value, the stronger the evidence that $A$ is superior to $B$ in that particular task, which can be interpreted as a confidence interval. Therefore, we perform comparisons between all methods for each optimization scenario outlined previously and for each dataset.

Our experiments uses common datasets used in Fair Classification problems, namely \textit{Adult Income}~\citep{misc_adult_2}, \textit{German Credit}~\citep{misc_statlog_(german_credit_data)_144}, \textit{Bank Marketing}~\citep{misc_bank_marketing_222}, and \textit{COMPAS Recidivism}~\citep{misc_compas}. We use the dataset readers available in the AI Fairness 360 toolkit~\citep{aif360-oct-2018} with its standard configurations. Instances with missing data are removed. To a proper description of those datasets please check Section~\ref{sec:experimental}

For all datasets, the data preparation process is the same, one-hot encoding for categorical features and standardize the numerical features. We perform a random split, with $80\%$ allocated for the hyperparameter tuning phase and the remaining $20\%$ reserved for evaluating metrics in the test phase. Within the hyperparameter tuning phase, this corresponding fraction of data is further randomly split, with $80\%$ assigned to training and $20\%$ to validation. The validation set allows us to assess metrics and compute the fitness function for each hyperparameter configuration. In datasets where there is originally some kind of split (e.g., train set and test set in separate files), all available data is merged and then shuffled to produce new splits at each run.

\section{Results and discussion}

Before presenting the main experimental results, we perform an additional comparison using the same methodology to address two concerns. Firstly, one might argue that the proposed penalty strategy is not distinctly effective compared to regular regularization factors like L2. In this context, any observed advantages of the proposed method over the same model without the penalty could be attributed to the conventional effects of regularization, i.e. avoiding model overfitting and consequently improving fitness values. Therefore, we compare the proposed strategy not only with an MLP baseline but also with the same MLP using standard L2 regularization across all network layers, optimized using the ranges specified in Table\ref{tab:hyperparameters_rpr}. 

Secondly, there is the question of whether using traditional correlation coefficients in order to assessing and penalizing redlining , such as Pearson's $\rho$, Spearman's $\rho_s$, and Kendall's $\tau$, would be equally beneficial. To address this, we compare the proposed penalty (referred as RPR$_\xi$) with variations that use the absolute values of Pearson's $\rho$, Spearman's $\rho_s$, and Kendall's $\tau$ instead of Chatterjee's $\xi$, referred to as RPR$\rho$, RPR${\rho_s}$, and RPR$\tau$, respectively. In this comparison, every method uses the same MLP architecture and the same hyperparameter tuning procedure, varying only the regularization strategy.

\begin{table}[ht]
    \centering
    \caption{Almost Stochastic Order test comparing RPR$_\xi$ fitness to baseline and multiple regularization schemes. Values under $0.5$ (in bold) mean that RPR$_\xi$ outperforms corresponding method in such optimization scenario.} \label{tab:aso_compare_rpr_variations}
    {\footnotesize
    \begin{tabular}{llccccc}
    \toprule
     \multirow{2}{*}{\shortstack[l]{Fairness/Performance\\Metric}} & Dataset & \multirow{2}{*}{\shortstack[c]{MLP\\(baseline)}} & \multirow{2}{*}{\shortstack[l]{MLP\\L2}} & \multirow{2}{*}{\shortstack[l]{MLP\\RPR$_{\rho}$}} & \multirow{2}{*}{\shortstack[l]{MLP\\RPR$_{\rho_s}$}} & \multirow{2}{*}{\shortstack[l]{MLP\\RPR$_{\tau}$}} \\ \\
\multirow{4}{*}{\shortstack[l]{Statistical Parity\\MCC}} 
 & Adult Income & 0.80 & 0.68 & 1.00 & 0.61 & 0.65 \\
 & Bank Marketing & 1.00 & \textbf{0.46} & 0.80 & 0.74 & 0.59 \\
 & COMPAS Recidivism & \textbf{0.02} & \textbf{0.17} & \textbf{0.18} & \textbf{0.15} & \textbf{0.09} \\
 & German Credit & 1.00 & 1.00 & 1.00 & \textbf{0.24} & 1.00 \\
\midrule
\multirow{4}{*}{\shortstack[l]{Equal Opportunity\\MCC}} 
 & Adult Income & \textbf{0.09} & \textbf{0.03} & \textbf{0.26} & \textbf{0.46} & \textbf{0.28} \\
 & Bank Marketing & 1.00 & 1.00 & 0.79 & 1.00 & 0.55 \\
 & COMPAS Recidivism & \textbf{0.19} & \textbf{0.23} & 0.60 & \textbf{0.33} & 0.89 \\
 & German Credit & 1.00 & 0.71 & 1.00 & 0.89 & 1.00 \\
\midrule
\multirow{4}{*}{\shortstack[l]{Equalized Odds\\MCC}} 
 & Adult Income & \textbf{0.29} & \textbf{0.24} & \textbf{0.25} & 0.63 & \textbf{0.46} \\
 & Bank Marketing & 1.00 & 1.00 & 1.00 & 1.00 & 1.00 \\
 & COMPAS Recidivism & \textbf{0.01} & \textbf{0.04} & \textbf{0.16} & \textbf{0.15} & \textbf{0.01} \\
 & German Credit & 1.00 & 0.63 & \textbf{0.38} & 1.00 & 1.00 \\
\midrule
\multirow{4}{*}{\shortstack[l]{Statistical Parity\\Accuracy}} 
 & Adult Income & 0.76 & \textbf{0.36} & \textbf{0.31} & \textbf{0.47} & \textbf{0.40} \\
 & Bank Marketing & 1.00 & 1.00 & 1.00 & 1.00 & 1.00 \\
 & COMPAS Recidivism & \textbf{0.42} & 0.60 & \textbf{0.48} & 1.00 & 0.72 \\
 & German Credit & 0.92 & 1.00 & 1.00 & 1.00 & 1.00 \\
\midrule
\multirow{4}{*}{\shortstack[l]{Equal Opportunity\\Accuracy}} 
 & Adult Income & \textbf{0.35} & \textbf{0.34} & \textbf{0.26} & 1.00 & \textbf{0.15} \\
 & Bank Marketing & 1.00 & 0.56 & 0.84 & 1.00 & 0.66 \\
 & COMPAS Recidivism & \textbf{0.02} & \textbf{0.04} & \textbf{0.06} & \textbf{0.10} & \textbf{0.19} \\
 & German Credit & 0.65 & 1.00 & 1.00 & 1.00 & 1.00 \\
\midrule
\multirow{4}{*}{\shortstack[l]{Equalized Odds\\Accuracy}} 
 & Adult Income & 1.00 & 0.74 & 0.68 & 0.53 & 0.66 \\
 & Bank Marketing & \textbf{0.19} & \textbf{0.15} & \textbf{0.21} & 0.52 & 0.55 \\
 & COMPAS Recidivism & \textbf{0.09} & \textbf{0.41} & \textbf{0.26} & \textbf{0.36} & \textbf{0.13} \\
 & German Credit & \textbf{0.26} & \textbf{0.11} & \textbf{0.50} & \textbf{0.25} & 1.00 \\
\bottomrule
\end{tabular}
    }
\end{table}

As we have multiple optimization scenarios with different objective functions and datasets, and to each of them multiple runs, we present in Table~\ref{tab:aso_compare_rpr_variations} the results of the ASO test described before, which allow us to properly compare each regularization strategy to RPR$_{\xi}$. Values under $0.5$ (in bold) mean that we can reject the null hypothesis, i.e., RPR$_{\xi}$ produces stochastically larger fitness than the regularization strategy in respective column for a objective and dataset. Lower values indicate stronger evidence. Additionally, complete results will be available at the end of this section, as like those presented to the Fair Transition Loss experiments.

Notably, both baseline MLP and MLP with L2 regularization performs similarly when compared to  RPR$_{\xi}$ on almost every scenario. This observation lead us to a comprehension that the proposed penalty schema differs from standard regularization strategies such as L2. Another relevant perception is that the proposed method is highly influenced by the dataset, which will be explored furthermore on this section.

Now we compare the results of the proposed penalty schema with standard correlation coefficients. We can perceive that in almost all scenarios where RPR$_{\xi}$ outperforms baseline MLP and MLP with L2 regularization it also outperforms the same penalty strategy but using  standard correlation coefficients instead of Chatterjee's. This enforces the argument that the proposed approach takes advantages of the Chatterjee's correlation coefficient characteristics. Despite that observation, in most of scenarios the ASO values are higher to those produced by baseline MLP and MLP with L2 regularization, indicating that they presents better performance/fairness trade-off. This can be interpret as an ability of the proposed regularization schema on penalizing redlining effect, although less efficiently than when using Chatterjee's correlation coefficient.

\begin{table}[ht]
\centering
\caption{Almost Stochastic Order test comparing the use of Redlining Penalty Regularizer on Standard MLP and MLP with Fair Transition Loss fitness. Values under $0.5$ (in bold) mean that the model with RPR outperforms the same model without regularization on each optimization scenario.} \label{tab:aso_compare_rpr}
{\footnotesize
\begin{tabular}{lrrrr|rrrr}
\toprule
\multirow{2}{*}{Fitness Rule} & \multicolumn{4}{c}{MLP} & \multicolumn{4}{c}{FTL} \\
& Adult & Bank & COMPAS & German & Adult & Bank & COMPAS & German \\
\midrule
MCC - S. Parity & 0.79 & 0.99 & \textbf{0.03} & 1.00 & \textbf{0.34} & 1.00 & 1.00 & \textbf{0.31} \\
MCC - Eq. Opp. & \textbf{0.26} & 1.00 & \textbf{0.01} & 1.00 & \textbf{0.35} & 1.00 & \textbf{0.19} & 1.00 \\
MCC - Eq. Odds & \textbf{0.08} & 1.00 & \textbf{0.20} & 1.00 & \textbf{0.30} & 1.00 & 0.56 & 1.00 \\
Acc. - S. Parity & 0.75 & 1.00 & \textbf{0.39} & 0.87 & \textbf{0.44} & 1.00 & \textbf{0.02} & 1.00 \\
Acc. - Eq. Opp. & 1.00 & \textbf{0.18} & \textbf{0.07} & \textbf{0.24} & 0.58 & 0.67 & 1.00 & 1.00 \\
Acc. - Eq. Odds & \textbf{0.33} & 1.00 & \textbf{0.01} & 0.62 & \textbf{0.40} & 1.00 & \textbf{0.20} & 1.00 \\
\bottomrule
\end{tabular}
}
\end{table}


\begin{figure}[!ht]
\centering
\caption{Fitness values of RPR optimizing MCC and multiple fairness metrics.}\label{fig:boxplot_mcc_rpr}
\begin{subfigure}{.32\linewidth}
    \caption{Statistical Parity}
    \label{fig:boxplot_mcc_parity_rpr}
    \includegraphics[width=1\linewidth]{images/boxplot_mcc_parity_rpr.pdf}
\end{subfigure}
\begin{subfigure}{.32\linewidth}
    \caption{Equal Opportunity}
    \label{fig:boxplot_mcc_opp_rpr}
    \includegraphics[width=1\linewidth]{images/boxplot_mcc_opportunity_rpr.pdf}
\end{subfigure}
\begin{subfigure}{.32\linewidth}
    \caption{Equalized Odds}
    \label{fig:boxplot_mcc_odds_rpr}
    \includegraphics[width=1\linewidth]{images/boxplot_mcc_odds_rpr.pdf}
\end{subfigure}
\end{figure}

\begin{figure}[!ht]
\centering
\caption{Fitness values of RPR optimizing Accuracy and multiple fairness metrics.}\label{fig:boxplot_acc_rpr}
\begin{subfigure}{.32\linewidth}
    \caption{Statistical Parity}
    \label{fig:boxplot_acc_parity_rpr}
    \includegraphics[width=1\linewidth]{images/boxplot_acc_parity_rpr.pdf}
\end{subfigure}
\begin{subfigure}{.32\linewidth}
    \caption{Equal Opportunity}
    \label{fig:boxplot_acc_opp_rpr}
    \includegraphics[width=1\linewidth]{images/boxplot_acc_opportunity_rpr.pdf}
\end{subfigure}
\begin{subfigure}{.32\linewidth}
    \caption{Equalized Odds}
    \label{fig:boxplot_acc_odds_rpr}
    \includegraphics[width=1\linewidth]{images/boxplot_acc_odds_rpr.pdf}
\end{subfigure}
\end{figure}

Now we compare the effects of Redlining Penalty Regularizer on both MLP with standard cross entropy loss and fair transition loss. The MLP baseline here is the same presented on the last comparison (Table~\ref{tab:aso_compare_rpr_variations}), and the FTL model uses the same MLP architecture. The overall methodology remains the same, including $15$ runs with dataset resampling with $100$ trials to hyperparameter tuning each, according Table~\ref{tab:hyperparameters_rpr}. The FTL model without RPR is exactly that same presented at Chapter~\ref{chap:ftl} and on \cite{Canalli2024}, whose achieved state-of-art results. In order to provide a straightforward comparison we include only RPR$_\xi$, as it reaches the best results when compared with the others. The ASO results are available at Table~\ref{tab:aso_compare_rpr} and the box-plot comparison at Figure~\ref{fig:boxplot_mcc_rpr} and Figure~\ref{fig:boxplot_acc_rpr}.

As an additional resource to interpret these results we plot at Figure~\ref{fig:datasets_correlation} the correlations of each feature to the sensitive to each described correlation coefficient at each dataset. The correlation values are presented from highest value to the lower, to provide a non-ascendant visualization. Also, we omit the correlation of sensitive feature to it self (always $1.0$) and plot both the top $10$ correlations and the values to each feature, according the number of features at corresponding dataset. Note that Chatterjee's performs very differently when compared to the others, identifying more highly correlation values, demonstrating the functional relationship of many features with the sensitive and therefore theirs potential to be used as proxy features by the model. Another relevant perception is that the \textit{Bank dataset} presents considerably lower correlations, although still very high values than the others.  

\begin{figure}[!ht]
\centering
\caption{Sorted feature's correlation to the sensitive one according multiple correlation coefficients to all datasets.}\label{fig:datasets_correlation}
\includegraphics[width=1\linewidth]{images/dataset_correlation_plots.pdf}
\end{figure}

On Table~\ref{tab:aso_compare_rpr} RPR presents low ASO values in many scenarios, on which we can claim that it outperforms the corresponding model without penalty. It is clear that both MLP and FTL benefit with the use of RPR, whose results remains strongly better to most of optimization objectives. Here the main difference is the dataset. While to \textit{Adult} and \textit{COMPAS} the model with RPR consistently outperforms its counterparts, this not happens when the proposed technique is evaluated on \textit{Bank} and \textit{German} datasets. As referred before, \textit{German Credit} is a very small and simple dataset. In this condition the Pareto frontier is rapidly achieved, with no room left for improvement. This condition clear looking at at Figure~\ref{fig:boxplot_mcc_rpr} and Figure~\ref{fig:boxplot_acc_rpr}.  

Additionally, when comparing \textit{Bank}'s features correlations to sensitive at Figure~\ref{fig:datasets_correlation} we can argue that as it presents reduced redlining effect, the effectiveness of the proposed penalty strategy is also reduced. Analogously, the best RPR results are those assessed on \textit{COMPAS}, the dataset with the most aggressive redlining effect.

Thus, on those conditions where the dataset presents features with high potential to be learned as proxy to the sensitive feature by the model, Redlining Penalty Regularizer consistently outperforms the model without regularization. This happens to both MLP models, the one using standard cross entropy loss and with Fair Transition Loss. In this last scenario the state-of-art results described before are enhanced.

As done on the last chapter, we present performance and fairness results along with the fitness, in order to provide material to a trade-off analysis. Those metric values enable the reader to compare this results with different experimental setup and fitness objective from literature. Results corresponding each optimization scenario can be found on tables \ref{tab:complete_mcc_parity_rpr} to \ref{tab:complete_acc_odds_rpr}, presenting metric means and standard deviation values across multiple resample run. Best result of each metric within evaluation scenario are in bold, and standard deviation values are presented between parenthesis. Up arrow ($\uparrow$) indicates that the referred metric should be maximized while down arrow ($\downarrow$) that the metric should be minimized. To provide a visual resource to this comparison, the distribution of those metrics across multiple resample runs comparing the Redlining Penalty Regularization using Chatterjee's correlation (FTL) on baseline (MLP) and Fair Transition Loss (FTL) are presented in joint plot format in figures \ref{fig:complete_mcc_parity_rpr} to \ref{fig:complete_acc_odds_rpr}. Each joint plot is present within the corresponding table to a better comprehension.

\newpage

\begin{table}
    \centering
    \caption{Mean and standard deviation metric values optimizing MCC and Statistical Parity in comparison with Redlining Penalty Regularizer.}\label{tab:complete_mcc_parity_rpr}
    {\scriptsize\begin{tabular}{llrrr}
    \toprule
    Dataset & Method & $\uparrow\;$Fitness & $\uparrow\;$MCC & $\downarrow\;$Stat. Parity \\
    \midrule
    \multirow{8}{*}{\shortstack[l]{Adult\\ Income}} & FTL & $0.487 \; (\pm0.02)$ & $0.509 \; (\pm0.02)$ & $0.022 \; (\pm0.02)$ \\
     & FTL+RPR$_{\xi}$ & $0.494 \; (\pm0.01)$ & $0.517 \; (\pm0.02)$ & $0.023 \; (\pm0.02)$ \\
     & MLP & $0.386 \; (\pm0.01)$ & $0.576 \; (\pm0.01)$ & $0.191 \; (\pm0.01)$ \\
     & MLP+L2 & $0.385 \; (\pm0.01)$ & $0.576 \; (\pm0.01)$ & $0.190 \; (\pm0.01)$ \\
     & MLP+RPR$_{\rho_s}$ & $0.384 \; (\pm0.01)$ & $0.576 \; (\pm0.01)$ & $0.192 \; (\pm0.01)$ \\
     & MLP+RPR$_{\rho}$ & $0.393 \; (\pm0.01)$ & $0.580 \; (\pm0.01)$ & $0.187 \; (\pm0.01)$ \\
     & MLP+RPR$_{\tau}$ & $0.386 \; (\pm0.01)$ & $0.577 \; (\pm0.01)$ & $0.191 \; (\pm0.01)$ \\
     & MLP+RPR$_{\xi}$ & $0.388 \; (\pm0.01)$ & $0.578 \; (\pm0.01)$ & $0.191 \; (\pm0.01)$ \\
    \midrule
    \multirow{8}{*}{\shortstack[l]{Bank\\ Marketing}} & FTL & $0.534 \; (\pm0.03)$ & $0.569 \; (\pm0.01)$ & $0.035 \; (\pm0.03)$ \\
     & FTL+RPR$_{\xi}$ & $0.523 \; (\pm0.04)$ & $0.569 \; (\pm0.01)$ & $0.046 \; (\pm0.04)$ \\
     & MLP & $0.429 \; (\pm0.03)$ & $0.521 \; (\pm0.02)$ & $0.092 \; (\pm0.02)$ \\
     & MLP+L2 & $0.412 \; (\pm0.04)$ & $0.521 \; (\pm0.02)$ & $0.110 \; (\pm0.03)$ \\
     & MLP+RPR$_{\rho_s}$ & $0.422 \; (\pm0.03)$ & $0.527 \; (\pm0.02)$ & $0.106 \; (\pm0.03)$ \\
     & MLP+RPR$_{\rho}$ & $0.424 \; (\pm0.03)$ & $0.523 \; (\pm0.01)$ & $0.100 \; (\pm0.03)$ \\
     & MLP+RPR$_{\tau}$ & $0.412 \; (\pm0.04)$ & $0.520 \; (\pm0.02)$ & $0.108 \; (\pm0.03)$ \\
     & MLP+RPR$_{\xi}$ & $0.426 \; (\pm0.05)$ & $0.528 \; (\pm0.02)$ & $0.102 \; (\pm0.04)$ \\
    \midrule
    \multirow{8}{*}{\shortstack[l]{Compas\\ Recidivism}} & FTL & $0.239 \; (\pm0.03)$ & $0.276 \; (\pm0.03)$ & $0.036 \; (\pm0.03)$ \\
     & FTL+RPR$_{\xi}$ & $0.236 \; (\pm0.05)$ & $0.294 \; (\pm0.03)$ & $0.058 \; (\pm0.04)$ \\
     & MLP & $0.074 \; (\pm0.03)$ & $0.283 \; (\pm0.02)$ & $0.209 \; (\pm0.04)$ \\
     & MLP+L2 & $0.089 \; (\pm0.04)$ & $0.291 \; (\pm0.02)$ & $0.202 \; (\pm0.04)$ \\
     & MLP+RPR$_{\rho_s}$ & $0.083 \; (\pm0.05)$ & $0.293 \; (\pm0.03)$ & $0.210 \; (\pm0.04)$ \\
     & MLP+RPR$_{\rho}$ & $0.087 \; (\pm0.05)$ & $0.282 \; (\pm0.03)$ & $0.195 \; (\pm0.04)$ \\
     & MLP+RPR$_{\tau}$ & $0.085 \; (\pm0.03)$ & $0.281 \; (\pm0.02)$ & $0.197 \; (\pm0.03)$ \\
     & MLP+RPR$_{\xi}$ & $0.121 \; (\pm0.05)$ & $0.329 \; (\pm0.03)$ & $0.208 \; (\pm0.03)$ \\
    \midrule
    \multirow{8}{*}{\shortstack[l]{German\\ Credit}} & FTL & $0.256 \; (\pm0.12)$ & $0.355 \; (\pm0.08)$ & $0.099 \; (\pm0.06)$ \\
     & FTL+RPR$_{\xi}$ & $0.302 \; (\pm0.06)$ & $0.371 \; (\pm0.05)$ & $0.069 \; (\pm0.05)$ \\
     & MLP & $0.266 \; (\pm0.10)$ & $0.329 \; (\pm0.09)$ & $0.064 \; (\pm0.05)$ \\
     & MLP+L2 & $0.261 \; (\pm0.10)$ & $0.374 \; (\pm0.09)$ & $0.113 \; (\pm0.07)$ \\
     & MLP+RPR$_{\rho_s}$ & $0.192 \; (\pm0.11)$ & $0.292 \; (\pm0.07)$ & $0.099 \; (\pm0.06)$ \\
     & MLP+RPR$_{\rho}$ & $0.255 \; (\pm0.07)$ & $0.341 \; (\pm0.07)$ & $0.087 \; (\pm0.04)$ \\
     & MLP+RPR$_{\tau}$ & $0.256 \; (\pm0.08)$ & $0.342 \; (\pm0.05)$ & $0.086 \; (\pm0.06)$ \\
     & MLP+RPR$_{\xi}$ & $0.246 \; (\pm0.07)$ & $0.329 \; (\pm0.05)$ & $0.084 \; (\pm0.06)$ \\
    \bottomrule
\end{tabular}}
\end{table}

 \begin{table}
    \centering
    \caption{Mean and standard deviation metric values optimizing MCC and Equal Opportunity in comparison with Redlining Penalty Regularizer.}\label{tab:complete_mcc_opportunity_rpr}
    {\scriptsize\begin{tabular}{llrrr}
    \toprule
    Dataset & Method & $\uparrow\;$Fitness & $\uparrow\;$MCC & $\downarrow\;$Eq. Opp. \\
    \midrule
    
    \multirow{8}{*}{\shortstack[l]{Adult\\ Income}} & FTL & $0.540 \; (\pm0.04)$ & $0.576 \; (\pm0.01)$ & $0.036 \; (\pm0.03)$ \\
     & FTL+RPR$_{\xi}$ & $0.554 \; (\pm0.03)$ & $0.581 \; (\pm0.02)$ & $0.027 \; (\pm0.02)$ \\
     & MLP & $0.480 \; (\pm0.04)$ & $0.582 \; (\pm0.01)$ & $0.103 \; (\pm0.04)$ \\
     & MLP+L2 & $0.479 \; (\pm0.03)$ & $0.580 \; (\pm0.01)$ & $0.102 \; (\pm0.03)$ \\
     & MLP+RPR$_{\rho_s}$ & $0.493 \; (\pm0.04)$ & $0.578 \; (\pm0.01)$ & $0.085 \; (\pm0.04)$ \\
     & MLP+RPR$_{\rho}$ & $0.478 \; (\pm0.04)$ & $0.580 \; (\pm0.01)$ & $0.102 \; (\pm0.04)$ \\
     & MLP+RPR$_{\tau}$ & $0.488 \; (\pm0.04)$ & $0.580 \; (\pm0.01)$ & $0.092 \; (\pm0.03)$ \\
     & MLP+RPR$_{\xi}$ & $0.506 \; (\pm0.05)$ & $0.577 \; (\pm0.01)$ & $0.071 \; (\pm0.05)$ \\
    \midrule
    \multirow{8}{*}{\shortstack[l]{Bank\\ Marketing}} & FTL & $0.483 \; (\pm0.07)$ & $0.567 \; (\pm0.02)$ & $0.084 \; (\pm0.06)$ \\
     & FTL+RPR$_{\xi}$ & $0.416 \; (\pm0.14)$ & $0.519 \; (\pm0.15)$ & $0.104 \; (\pm0.08)$ \\
     & MLP & $0.420 \; (\pm0.08)$ & $0.524 \; (\pm0.02)$ & $0.104 \; (\pm0.08)$ \\
     & MLP+L2 & $0.438 \; (\pm0.06)$ & $0.514 \; (\pm0.02)$ & $0.076 \; (\pm0.06)$ \\
     & MLP+RPR$_{\rho_s}$ & $0.426 \; (\pm0.06)$ & $0.520 \; (\pm0.02)$ & $0.094 \; (\pm0.06)$ \\
     & MLP+RPR$_{\rho}$ & $0.452 \; (\pm0.05)$ & $0.527 \; (\pm0.02)$ & $0.074 \; (\pm0.05)$ \\
     & MLP+RPR$_{\tau}$ & $0.417 \; (\pm0.08)$ & $0.526 \; (\pm0.02)$ & $0.109 \; (\pm0.07)$ \\
     & MLP+RPR$_{\xi}$ & $0.420 \; (\pm0.10)$ & $0.529 \; (\pm0.02)$ & $0.109 \; (\pm0.09)$ \\
    \midrule
    \multirow{8}{*}{\shortstack[l]{COMPAS\\ Recidivism}} & FTL & $0.195 \; (\pm0.11)$ & $0.281 \; (\pm0.03)$ & $0.086 \; (\pm0.09)$ \\
     & FTL+RPR$_{\xi}$ & $0.251 \; (\pm0.04)$ & $0.309 \; (\pm0.03)$ & $0.058 \; (\pm0.04)$ \\
     & MLP & $0.150 \; (\pm0.05)$ & $0.282 \; (\pm0.03)$ & $0.132 \; (\pm0.05)$ \\
     & MLP+L2 & $0.146 \; (\pm0.06)$ & $0.282 \; (\pm0.03)$ & $0.136 \; (\pm0.04)$ \\
     & MLP+RPR$_{\rho_s}$ & $0.179 \; (\pm0.04)$ & $0.303 \; (\pm0.02)$ & $0.124 \; (\pm0.03)$ \\
     & MLP+RPR$_{\rho}$ & $0.168 \; (\pm0.06)$ & $0.290 \; (\pm0.03)$ & $0.122 \; (\pm0.05)$ \\
     & MLP+RPR$_{\tau}$ & $0.146 \; (\pm0.05)$ & $0.289 \; (\pm0.03)$ & $0.143 \; (\pm0.04)$ \\
     & MLP+RPR$_{\xi}$ & $0.211 \; (\pm0.05)$ & $0.325 \; (\pm0.02)$ & $0.114 \; (\pm0.04)$ \\
    \midrule
    \multirow{8}{*}{\shortstack[l]{German\\ Credit}} & FTL & $0.293 \; (\pm0.12)$ & $0.368 \; (\pm0.10)$ & $0.074 \; (\pm0.05)$ \\
     & FTL+RPR$_{\xi}$ & $0.166 \; (\pm0.16)$ & $0.284 \; (\pm0.14)$ & $0.117 \; (\pm0.07)$ \\
     & MLP & $0.290 \; (\pm0.09)$ & $0.355 \; (\pm0.07)$ & $0.065 \; (\pm0.06)$ \\
     & MLP+L2 & $0.249 \; (\pm0.08)$ & $0.309 \; (\pm0.07)$ & $0.060 \; (\pm0.04)$ \\
     & MLP+RPR$_{\rho_s}$ & $0.319 \; (\pm0.05)$ & $0.367 \; (\pm0.07)$ & $0.048 \; (\pm0.04)$ \\
     & MLP+RPR$_{\rho}$ & $0.229 \; (\pm0.13)$ & $0.297 \; (\pm0.10)$ & $0.068 \; (\pm0.04)$ \\
     & MLP+RPR$_{\tau}$ & $0.277 \; (\pm0.08)$ & $0.329 \; (\pm0.06)$ & $0.053 \; (\pm0.04)$ \\
     & MLP+RPR$_{\xi}$ & $0.275 \; (\pm0.09)$ & $0.341 \; (\pm0.06)$ & $0.066 \; (\pm0.05)$ \\
     \bottomrule
\end{tabular}}
\end{table}
 \begin{table}
    \centering
    \caption{Mean and standard deviation metric values optimizing MCC and Equalized Odds in comparison with Redlining Penalty Regularizer.}\label{tab:complete_mcc_odds_rpr}
    {\scriptsize\begin{tabular}{llrrr}
    \toprule
    Dataset & Method & $\uparrow\;$Fitness & $\uparrow\;$MCC & $\downarrow\;$Eq. Odds \\
    \midrule
    
    \multirow{8}{*}{\shortstack[l]{Adult\\ Income}} & FTL & $0.515 \; (\pm0.03)$ & $0.572 \; (\pm0.02)$ & $0.057 \; (\pm0.02)$ \\
     & FTL+RPR$_{\xi}$ & $0.526 \; (\pm0.02)$ & $0.575 \; (\pm0.01)$ & $0.048 \; (\pm0.02)$ \\
     & MLP & $0.478 \; (\pm0.02)$ & $0.575 \; (\pm0.01)$ & $0.097 \; (\pm0.02)$ \\
     & MLP+L2 & $0.484 \; (\pm0.01)$ & $0.577 \; (\pm0.01)$ & $0.093 \; (\pm0.01)$ \\
     & MLP+RPR$_{\rho_s}$ & $0.492 \; (\pm0.02)$ & $0.578 \; (\pm0.01)$ & $0.085 \; (\pm0.02)$ \\
     & MLP+RPR$_{\rho}$ & $0.489 \; (\pm0.02)$ & $0.575 \; (\pm0.01)$ & $0.087 \; (\pm0.02)$ \\
     & MLP+RPR$_{\tau}$ & $0.487 \; (\pm0.02)$ & $0.578 \; (\pm0.01)$ & $0.091 \; (\pm0.02)$ \\
     & MLP+RPR$_{\xi}$ & $0.500 \; (\pm0.02)$ & $0.578 \; (\pm0.01)$ & $0.078 \; (\pm0.02)$ \\
    \midrule
    \multirow{8}{*}{\shortstack[l]{Bank\\ Marketing}} & FTL & $0.495 \; (\pm0.05)$ & $0.574 \; (\pm0.01)$ & $0.078 \; (\pm0.05)$ \\
     & FTL+RPR$_{\xi}$ & $0.490 \; (\pm0.03)$ & $0.571 \; (\pm0.01)$ & $0.081 \; (\pm0.03)$ \\
     & MLP & $0.461 \; (\pm0.05)$ & $0.522 \; (\pm0.02)$ & $0.061 \; (\pm0.04)$ \\
     & MLP+L2 & $0.461 \; (\pm0.03)$ & $0.525 \; (\pm0.02)$ & $0.065 \; (\pm0.03)$ \\
     & MLP+RPR$_{\rho_s}$ & $0.468 \; (\pm0.04)$ & $0.529 \; (\pm0.02)$ & $0.061 \; (\pm0.03)$ \\
     & MLP+RPR$_{\rho}$ & $0.442 \; (\pm0.05)$ & $0.522 \; (\pm0.02)$ & $0.080 \; (\pm0.03)$ \\
     & MLP+RPR$_{\tau}$ & $0.434 \; (\pm0.05)$ & $0.519 \; (\pm0.01)$ & $0.085 \; (\pm0.05)$ \\
     & MLP+RPR$_{\xi}$ & $0.446 \; (\pm0.04)$ & $0.539 \; (\pm0.02)$ & $0.093 \; (\pm0.04)$ \\
    \midrule
    \multirow{8}{*}{\shortstack[l]{COMPAS\\ Recidivism}} & FTL & $0.216 \; (\pm0.04)$ & $0.288 \; (\pm0.02)$ & $0.072 \; (\pm0.04)$ \\
     & FTL+RPR$_{\xi}$ & $0.225 \; (\pm0.04)$ & $0.282 \; (\pm0.03)$ & $0.056 \; (\pm0.03)$ \\
     & MLP & $0.098 \; (\pm0.05)$ & $0.283 \; (\pm0.03)$ & $0.185 \; (\pm0.03)$ \\
     & MLP+L2 & $0.097 \; (\pm0.04)$ & $0.281 \; (\pm0.03)$ & $0.185 \; (\pm0.04)$ \\
     & MLP+RPR$_{\rho_s}$ & $0.096 \; (\pm0.05)$ & $0.282 \; (\pm0.03)$ & $0.185 \; (\pm0.03)$ \\
     & MLP+RPR$_{\rho}$ & $0.119 \; (\pm0.05)$ & $0.292 \; (\pm0.03)$ & $0.174 \; (\pm0.03)$ \\
     & MLP+RPR$_{\tau}$ & $0.124 \; (\pm0.04)$ & $0.310 \; (\pm0.02)$ & $0.185 \; (\pm0.03)$ \\
     & MLP+RPR$_{\xi}$ & $0.129 \; (\pm0.05)$ & $0.303 \; (\pm0.02)$ & $0.174 \; (\pm0.03)$ \\
    \midrule
    \multirow{8}{*}{\shortstack[l]{German\\ Credit}} & FTL & $0.278 \; (\pm0.12)$ & $0.383 \; (\pm0.08)$ & $0.105 \; (\pm0.06)$ \\
     & FTL+RPR$_{\xi}$ & $0.228 \; (\pm0.08)$ & $0.337 \; (\pm0.07)$ & $0.109 \; (\pm0.06)$ \\
     & MLP & $0.249 \; (\pm0.10)$ & $0.352 \; (\pm0.08)$ & $0.102 \; (\pm0.06)$ \\
     & MLP+L2 & $0.214 \; (\pm0.11)$ & $0.297 \; (\pm0.10)$ & $0.084 \; (\pm0.05)$ \\
     & MLP+RPR$_{\rho_s}$ & $0.219 \; (\pm0.09)$ & $0.352 \; (\pm0.07)$ & $0.133 \; (\pm0.05)$ \\
     & MLP+RPR$_{\rho}$ & $0.228 \; (\pm0.11)$ & $0.342 \; (\pm0.06)$ & $0.114 \; (\pm0.09)$ \\
     & MLP+RPR$_{\tau}$ & $0.247 \; (\pm0.06)$ & $0.341 \; (\pm0.06)$ & $0.094 \; (\pm0.05)$ \\
     & MLP+RPR$_{\xi}$ & $0.231 \; (\pm0.10)$ & $0.332 \; (\pm0.07)$ & $0.102 \; (\pm0.06)$ \\
     \bottomrule
\end{tabular}}
\end{table}

 \begin{table}
    \centering
    \caption{Mean and standard deviation metric values optimizing Accuracy and Statistical Parity in comparison with Redlining Penalty Regularizer.}\label{tab:complete_acc_parity_rpr}
   {\scriptsize \begin{tabular}{llrrr}
    \toprule
    Dataset & Method & $\uparrow\;$Fitness & $\uparrow\;$Accuracy & $\downarrow\;$Stat. Parity \\
    \midrule
    
    \multirow{8}{*}{\shortstack[l]{Adult\\ Income}} & FTL & $0.806 \; (\pm0.02)$ & $0.827 \; (\pm0.01)$ & $0.022 \; (\pm0.02)$ \\
     & FTL+RPR$_{\xi}$ & $0.811 \; (\pm0.01)$ & $0.825 \; (\pm0.01)$ & $0.015 \; (\pm0.01)$ \\
     & MLP & $0.664 \; (\pm0.01)$ & $0.850 \; (\pm0.00)$ & $0.185 \; (\pm0.01)$ \\
     & MLP+L2 & $0.660 \; (\pm0.01)$ & $0.851 \; (\pm0.00)$ & $0.191 \; (\pm0.01)$ \\
     & MLP+RPR$_{\rho_s}$ & $0.663 \; (\pm0.01)$ & $0.848 \; (\pm0.00)$ & $0.186 \; (\pm0.01)$ \\
     & MLP+RPR$_{\rho}$ & $0.657 \; (\pm0.01)$ & $0.849 \; (\pm0.00)$ & $0.192 \; (\pm0.01)$ \\
     & MLP+RPR$_{\tau}$ & $0.659 \; (\pm0.01)$ & $0.849 \; (\pm0.00)$ & $0.190 \; (\pm0.01)$ \\
     & MLP+RPR$_{\xi}$ & $0.666 \; (\pm0.01)$ & $0.850 \; (\pm0.00)$ & $0.183 \; (\pm0.02)$ \\
    \midrule
    \multirow{8}{*}{\shortstack[l]{Bank\\ Marketing}} & FTL & $0.828 \; (\pm0.14)$ & $0.887 \; (\pm0.01)$ & $0.059 \; (\pm0.14)$ \\
     & FTL+RPR$_{\xi}$ & $0.800 \; (\pm0.25)$ & $0.897 \; (\pm0.01)$ & $0.096 \; (\pm0.24)$ \\
     & MLP & $0.799 \; (\pm0.03)$ & $0.902 \; (\pm0.00)$ & $0.103 \; (\pm0.03)$ \\
     & MLP+L2 & $0.796 \; (\pm0.03)$ & $0.902 \; (\pm0.00)$ & $0.106 \; (\pm0.03)$ \\
     & MLP+RPR$_{\rho_s}$ & $0.790 \; (\pm0.03)$ & $0.902 \; (\pm0.00)$ & $0.113 \; (\pm0.03)$ \\
     & MLP+RPR$_{\rho}$ & $0.799 \; (\pm0.03)$ & $0.902 \; (\pm0.00)$ & $0.103 \; (\pm0.03)$ \\
     & MLP+RPR$_{\tau}$ & $0.802 \; (\pm0.05)$ & $0.901 \; (\pm0.00)$ & $0.099 \; (\pm0.05)$ \\
     & MLP+RPR$_{\xi}$ & $0.777 \; (\pm0.03)$ & $0.905 \; (\pm0.00)$ & $0.127 \; (\pm0.04)$ \\
    \midrule
    \multirow{8}{*}{\shortstack[l]{COMPAS\\ Recidivism}} & FTL & $0.520 \; (\pm0.12)$ & $0.619 \; (\pm0.04)$ & $0.100 \; (\pm0.11)$ \\
     & FTL+RPR$_{\xi}$ & $0.606 \; (\pm0.03)$ & $0.651 \; (\pm0.01)$ & $0.045 \; (\pm0.03)$ \\
     & MLP & $0.439 \; (\pm0.04)$ & $0.646 \; (\pm0.01)$ & $0.207 \; (\pm0.04)$ \\
     & MLP+L2 & $0.449 \; (\pm0.04)$ & $0.650 \; (\pm0.01)$ & $0.202 \; (\pm0.04)$ \\
     & MLP+RPR$_{\rho_s}$ & $0.462 \; (\pm0.04)$ & $0.651 \; (\pm0.01)$ & $0.189 \; (\pm0.04)$ \\
     & MLP+RPR$_{\rho}$ & $0.449 \; (\pm0.03)$ & $0.652 \; (\pm0.01)$ & $0.203 \; (\pm0.03)$ \\
     & MLP+RPR$_{\tau}$ & $0.457 \; (\pm0.03)$ & $0.646 \; (\pm0.01)$ & $0.189 \; (\pm0.03)$ \\
     & MLP+RPR$_{\xi}$ & $0.462 \; (\pm0.03)$ & $0.659 \; (\pm0.01)$ & $0.197 \; (\pm0.03)$ \\
    \midrule
    \multirow{8}{*}{\shortstack[l]{German\\ Credit}} & FTL & $0.684 \; (\pm0.07)$ & $0.723 \; (\pm0.03)$ & $0.040 \; (\pm0.05)$ \\
     & FTL+RPR$_{\xi}$ & $0.655 \; (\pm0.07)$ & $0.725 \; (\pm0.03)$ & $0.070 \; (\pm0.06)$ \\
     & MLP & $0.628 \; (\pm0.06)$ & $0.742 \; (\pm0.03)$ & $0.114 \; (\pm0.06)$ \\
     & MLP+L2 & $0.653 \; (\pm0.08)$ & $0.741 \; (\pm0.03)$ & $0.088 \; (\pm0.07)$ \\
     & MLP+RPR$_{\rho_s}$ & $0.644 \; (\pm0.05)$ & $0.742 \; (\pm0.03)$ & $0.098 \; (\pm0.04)$ \\
     & MLP+RPR$_{\rho}$ & $0.669 \; (\pm0.06)$ & $0.763 \; (\pm0.02)$ & $0.094 \; (\pm0.05)$ \\
     & MLP+RPR$_{\tau}$ & $0.650 \; (\pm0.06)$ & $0.747 \; (\pm0.03)$ & $0.097 \; (\pm0.05)$ \\
     & MLP+RPR$_{\xi}$ & $0.631 \; (\pm0.07)$ & $0.733 \; (\pm0.02)$ & $0.102 \; (\pm0.06)$ \\
     \bottomrule
\end{tabular}}
\end{table}


\begin{table}
    \centering
    \caption{Mean and standard deviation metric values optimizing Accuracy and Equal Opportunity in comparison with Redlining Penalty Regularizer.}\label{tab:complete_acc_opportunity_rpr}
    {\scriptsize \begin{tabular}{llrrr}
    \toprule
    Dataset & Method & $\uparrow\;$Fitness & $\uparrow\;$Accuracy & $\downarrow\;$Eq. Opp. \\
    \midrule
        
    \multirow{8}{*}{\shortstack[l]{Adult\\ Income}} & FTL & $0.812 \; (\pm0.03)$ & $0.845 \; (\pm0.01)$ & $0.034 \; (\pm0.02)$ \\
     & FTL+RPR$_{\xi}$ & $0.815 \; (\pm0.02)$ & $0.847 \; (\pm0.00)$ & $0.031 \; (\pm0.02)$ \\
     & MLP & $0.758 \; (\pm0.04)$ & $0.848 \; (\pm0.00)$ & $0.090 \; (\pm0.04)$ \\
     & MLP+L2 & $0.750 \; (\pm0.04)$ & $0.850 \; (\pm0.00)$ & $0.100 \; (\pm0.04)$ \\
     & MLP+RPR$_{\rho_s}$ & $0.748 \; (\pm0.03)$ & $0.849 \; (\pm0.00)$ & $0.101 \; (\pm0.03)$ \\
     & MLP+RPR$_{\rho}$ & $0.750 \; (\pm0.04)$ & $0.848 \; (\pm0.00)$ & $0.098 \; (\pm0.04)$ \\
     & MLP+RPR$_{\tau}$ & $0.751 \; (\pm0.03)$ & $0.849 \; (\pm0.00)$ & $0.098 \; (\pm0.03)$ \\
     & MLP+RPR$_{\xi}$ & $0.757 \; (\pm0.05)$ & $0.849 \; (\pm0.00)$ & $0.091 \; (\pm0.05)$ \\
    \midrule
    \multirow{8}{*}{\shortstack[l]{Bank\\ Marketing}} & FTL & $0.781 \; (\pm0.17)$ & $0.883 \; (\pm0.02)$ & $0.102 \; (\pm0.17)$ \\
     & FTL+RPR$_{\xi}$ & $0.802 \; (\pm0.08)$ & $0.892 \; (\pm0.01)$ & $0.090 \; (\pm0.09)$ \\
     & MLP & $0.803 \; (\pm0.07)$ & $0.902 \; (\pm0.00)$ & $0.099 \; (\pm0.07)$ \\
     & MLP+L2 & $0.791 \; (\pm0.08)$ & $0.903 \; (\pm0.00)$ & $0.112 \; (\pm0.07)$ \\
     & MLP+RPR$_{\rho_s}$ & $0.824 \; (\pm0.06)$ & $0.902 \; (\pm0.00)$ & $0.078 \; (\pm0.06)$ \\
     & MLP+RPR$_{\rho}$ & $0.798 \; (\pm0.07)$ & $0.901 \; (\pm0.00)$ & $0.103 \; (\pm0.07)$ \\
     & MLP+RPR$_{\tau}$ & $0.822 \; (\pm0.07)$ & $0.901 \; (\pm0.00)$ & $0.079 \; (\pm0.07)$ \\
     & MLP+RPR$_{\xi}$ & $0.839 \; (\pm0.05)$ & $0.903 \; (\pm0.00)$ & $0.063 \; (\pm0.04)$ \\
    \midrule
    \multirow{8}{*}{\shortstack[l]{COMPAS\\ Recidivism}} & FTL & $0.614 \; (\pm0.03)$ & $0.645 \; (\pm0.03)$ & $0.031 \; (\pm0.02)$ \\
     & FTL+RPR$_{\xi}$ & $0.598 \; (\pm0.03)$ & $0.639 \; (\pm0.02)$ & $0.041 \; (\pm0.03)$ \\
     & MLP & $0.498 \; (\pm0.04)$ & $0.646 \; (\pm0.01)$ & $0.148 \; (\pm0.04)$ \\
     & MLP+L2 & $0.519 \; (\pm0.04)$ & $0.647 \; (\pm0.02)$ & $0.128 \; (\pm0.03)$ \\
     & MLP+RPR$_{\rho_s}$ & $0.520 \; (\pm0.03)$ & $0.651 \; (\pm0.01)$ & $0.131 \; (\pm0.03)$ \\
     & MLP+RPR$_{\rho}$ & $0.503 \; (\pm0.06)$ & $0.649 \; (\pm0.01)$ & $0.146 \; (\pm0.05)$ \\
     & MLP+RPR$_{\tau}$ & $0.506 \; (\pm0.04)$ & $0.647 \; (\pm0.01)$ & $0.141 \; (\pm0.04)$ \\
     & MLP+RPR$_{\xi}$ & $0.536 \; (\pm0.03)$ & $0.663 \; (\pm0.01)$ & $0.127 \; (\pm0.03)$ \\
    \midrule
    \multirow{8}{*}{\shortstack[l]{German\\ Credit}} & FTL & $0.676 \; (\pm0.06)$ & $0.751 \; (\pm0.02)$ & $0.075 \; (\pm0.06)$ \\
     & FTL+RPR$_{\xi}$ & $0.665 \; (\pm0.06)$ & $0.736 \; (\pm0.03)$ & $0.071 \; (\pm0.05)$ \\
     & MLP & $0.673 \; (\pm0.06)$ & $0.739 \; (\pm0.03)$ & $0.066 \; (\pm0.05)$ \\
     & MLP+L2 & $0.652 \; (\pm0.04)$ & $0.736 \; (\pm0.03)$ & $0.084 \; (\pm0.05)$ \\
     & MLP+RPR$_{\rho_s}$ & $0.667 \; (\pm0.06)$ & $0.739 \; (\pm0.03)$ & $0.072 \; (\pm0.05)$ \\
     & MLP+RPR$_{\rho}$ & $0.687 \; (\pm0.04)$ & $0.749 \; (\pm0.04)$ & $0.062 \; (\pm0.03)$ \\
     & MLP+RPR$_{\tau}$ & $0.702 \; (\pm0.04)$ & $0.736 \; (\pm0.02)$ & $0.034 \; (\pm0.03)$ \\
     & MLP+RPR$_{\xi}$ & $0.702 \; (\pm0.05)$ & $0.750 \; (\pm0.02)$ & $0.048 \; (\pm0.04)$ \\
     \bottomrule
\end{tabular} }
\end{table}

 \begin{table}
    \centering
    \caption{Mean and standard deviation metric values optimizing Accuracy and Equalized Odds in comparison with Redlining Penalty Regularizer.}\label{tab:complete_acc_odds_rpr}
    {\scriptsize \begin{tabular}{llrrr}
    \toprule
    Dataset & Method & $\uparrow\;$Fitness & $\uparrow\;$Accuracy & $\downarrow\;$Eq. Odds \\
    \midrule
        
    \multirow{8}{*}{\shortstack[l]{Adult\\ Income}} & FTL & $0.798 \; (\pm0.02)$ & $0.841 \; (\pm0.01)$ & $0.043 \; (\pm0.02)$ \\
     & FTL+RPR$_{\xi}$ & $0.805 \; (\pm0.02)$ & $0.843 \; (\pm0.01)$ & $0.038 \; (\pm0.02)$ \\
     & MLP & $0.760 \; (\pm0.02)$ & $0.849 \; (\pm0.00)$ & $0.089 \; (\pm0.02)$ \\
     & MLP+L2 & $0.762 \; (\pm0.02)$ & $0.849 \; (\pm0.00)$ & $0.086 \; (\pm0.02)$ \\
     & MLP+RPR$_{\rho_s}$ & $0.772 \; (\pm0.02)$ & $0.849 \; (\pm0.00)$ & $0.076 \; (\pm0.02)$ \\
     & MLP+RPR$_{\rho}$ & $0.759 \; (\pm0.02)$ & $0.849 \; (\pm0.00)$ & $0.090 \; (\pm0.02)$ \\
     & MLP+RPR$_{\tau}$ & $0.753 \; (\pm0.03)$ & $0.849 \; (\pm0.00)$ & $0.096 \; (\pm0.02)$ \\
     & MLP+RPR$_{\xi}$ & $0.772 \; (\pm0.02)$ & $0.849 \; (\pm0.00)$ & $0.077 \; (\pm0.02)$ \\
    \midrule
    \multirow{8}{*}{\shortstack[l]{Bank\\ Marketing}} & FTL & $0.846 \; (\pm0.03)$ & $0.890 \; (\pm0.01)$ & $0.044 \; (\pm0.04)$ \\
     & FTL+RPR$_{\xi}$ & $0.831 \; (\pm0.05)$ & $0.892 \; (\pm0.01)$ & $0.061 \; (\pm0.05)$ \\
     & MLP & $0.845 \; (\pm0.03)$ & $0.901 \; (\pm0.00)$ & $0.057 \; (\pm0.03)$ \\
     & MLP+L2 & $0.821 \; (\pm0.04)$ & $0.903 \; (\pm0.00)$ & $0.082 \; (\pm0.04)$ \\
     & MLP+RPR$_{\rho_s}$ & $0.838 \; (\pm0.04)$ & $0.903 \; (\pm0.00)$ & $0.065 \; (\pm0.04)$ \\
     & MLP+RPR$_{\rho}$ & $0.828 \; (\pm0.05)$ & $0.902 \; (\pm0.00)$ & $0.073 \; (\pm0.05)$ \\
     & MLP+RPR$_{\tau}$ & $0.822 \; (\pm0.04)$ & $0.902 \; (\pm0.00)$ & $0.079 \; (\pm0.04)$ \\
     & MLP+RPR$_{\xi}$ & $0.830 \; (\pm0.03)$ & $0.902 \; (\pm0.00)$ & $0.072 \; (\pm0.03)$ \\
    \midrule
    \multirow{8}{*}{\shortstack[l]{COMPAS\\ Recidivism}} & FTL & $0.545 \; (\pm0.10)$ & $0.631 \; (\pm0.05)$ & $0.086 \; (\pm0.08)$ \\
     & FTL+RPR$_{\xi}$ & $0.594 \; (\pm0.04)$ & $0.647 \; (\pm0.02)$ & $0.053 \; (\pm0.04)$ \\
     & MLP & $0.449 \; (\pm0.04)$ & $0.649 \; (\pm0.01)$ & $0.200 \; (\pm0.03)$ \\
     & MLP+L2 & $0.452 \; (\pm0.04)$ & $0.649 \; (\pm0.01)$ & $0.197 \; (\pm0.04)$ \\
     & MLP+RPR$_{\rho_s}$ & $0.464 \; (\pm0.03)$ & $0.650 \; (\pm0.01)$ & $0.186 \; (\pm0.03)$ \\
     & MLP+RPR$_{\rho}$ & $0.449 \; (\pm0.05)$ & $0.650 \; (\pm0.01)$ & $0.201 \; (\pm0.04)$ \\
     & MLP+RPR$_{\tau}$ & $0.463 \; (\pm0.05)$ & $0.650 \; (\pm0.01)$ & $0.187 \; (\pm0.05)$ \\
     & MLP+RPR$_{\xi}$ & $0.497 \; (\pm0.04)$ & $0.667 \; (\pm0.01)$ & $0.170 \; (\pm0.03)$ \\
    \midrule
    \multirow{8}{*}{\shortstack[l]{German\\ Credit}} & FTL & $0.669 \; (\pm0.05)$ & $0.712 \; (\pm0.02)$ & $0.043 \; (\pm0.07)$ \\
     & FTL+RPR$_{\xi}$ & $0.631 \; (\pm0.06)$ & $0.721 \; (\pm0.03)$ & $0.090 \; (\pm0.08)$ \\
     & MLP & $0.619 \; (\pm0.07)$ & $0.740 \; (\pm0.03)$ & $0.121 \; (\pm0.07)$ \\
     & MLP+L2 & $0.675 \; (\pm0.05)$ & $0.750 \; (\pm0.02)$ & $0.075 \; (\pm0.04)$ \\
     & MLP+RPR$_{\rho_s}$ & $0.647 \; (\pm0.06)$ & $0.742 \; (\pm0.03)$ & $0.096 \; (\pm0.05)$ \\
     & MLP+RPR$_{\rho}$ & $0.641 \; (\pm0.04)$ & $0.734 \; (\pm0.03)$ & $0.093 \; (\pm0.04)$ \\
     & MLP+RPR$_{\tau}$ & $0.647 \; (\pm0.07)$ & $0.748 \; (\pm0.03)$ & $0.101 \; (\pm0.07)$ \\
     & MLP+RPR$_{\xi}$ & $0.640 \; (\pm0.06)$ & $0.748 \; (\pm0.02)$ & $0.107 \; (\pm0.06)$ \\
     \bottomrule
\end{tabular}}
\end{table}
  \chapter{Conclusions}\label{chap:conclusions}


%\section{Considerations on the proposal}

In this work we introduced Fair Transition Loss, a novel approach to fair classification that estimates the influence of historical and societal biases on outcome probabilities for distinct social groups delineated by a sensitive feature. Drawing inspiration from label noise robustness, we represented these disparities on model's positive and negative outcomes probabilities to each social group using transition matrices, therefore incorporating this information onto the loss function to promote fairness. The proposed method hyperarameters were chosen by a Multi-Objective Optimization approach combining both fairness and model performance with a linear scalarization, defined in such a way that it is suitable to optimize a wide range of fairness and performance metrics.

Also, the present study proposed a novel regularization approach to fair classification named Redlining Penalty Regularization, which uses feature's correlation coefficient to the sensitive attribute to proportionately penalize model's dependency on it. Our empirical evaluation demonstrated that this approach effectively mitigate unfairness while keeping predictive performance, with benefits corresponding to redlining level on dataset. The proposed approach can be used on both standard neural networks and those trained with Fair Transition Loss to reduce bias while keeping predictive performance.

\section{Results and contributions}

Our experimental evaluation indicates that Fair Transition Loss consistently outperforms its competitors in most optimization scenarios. Even in those cases that the proposed method isn't the outright leader, it performs at least as well as evaluated alternatives, standing as the only model to keep competitive results in all scenarios. Therefore, this novel approach can significantly mitigate bias while keeping model performance, specially when optimizing balanced performance metrics like MCC. The proposed technique particularly stands out in setups where hyperparameter tuning procedures constitutes the prediction pipeline.

Furthermore, our results indicates that the Redlining Penalty Regularization approach effectively mitigates redlining effect on multiple datasets within various objective metrics when applied to both standard MLP and MLP trained with Fair Transition Loss. Also, our analysis indicates that the effectiveness of the referred approach is proportional to redlining level present on data, the higher the redlining the higher the performance-fairness trade-off improvement.

Thus, we summarize some contributions of the present study:

\begin{enumerate}[(i)]
    \item a novel loss correction approach inspired by label noise techniques to fair classification problems;
    \item a discussion of recent studies that lies between fairness and noise on machine learning;
    \item a multi-objective hyperparameter tuning approach do tackle the performance-fairness trade-off using a simple linear scalarization setup;
    \item a solid comparison of classic and state-of-art fair classification approaches using the Almost Stochastic Order as significance test;
    \item state-of-art results on various benchmarked datasets to fair classification using the proposed loss correction approach;
    \item a novel regularization approach that proportionately penalizes model's dependency on sensitive feature proxies according their correlations;
    \item improved results using the proposed regularization approach on standard MLPs to fair classification.
    \item improved state-of-art results using both the loss correction and regularization approaches to fair classification.
\end{enumerate}

\section{Research directions}

Here we outline some research direction insights, derived from proposed method's drawbacks and issues uncharted by this study:

\begin{enumerate}[(i)]
    \item explore approaches to estimating or initializing transition matrices to reduce computational costs required by hyperparameter optimization techniques;
    \item evaluate Fair Transition Loss within different neural network architectures, such as Deep Neural Networks;
    \item evaluate Fair Transition Loss on different data domains, such as image, audio and natural language;
    \item evaluate Fair Transition Loss optimization under non-linear scalarization setups, such as the Chebyshev scalarization scheme proposed by~\cite{Wei2022};
    \item evaluate Fair Transition Loss within different multi-objective optimization schemes, such as the Fair Hyperparameter Tuning techniques proposed by~\cite{Cruz2021};
    \item investigate whether Fair Transition Loss can effectively address multi-class fair classification problems and handle multiple sensitive attributes, as theoretically possible;
    \item evaluate Redlining Penalty Regularization within different neural network architectures, such as Deep Neural Networks;
    \item evaluate Redlining Penalty Regularization on different data domains, such as image, audio and natural language;
    \item evaluate Redlining Penalty Regularization combined with multiple pre-processing, in-processing and post-processing fair classification approaches.
    
\end{enumerate}



 
  %\chapter{Proposta}
  %\section{Ruído e Injustiça}
  %\section{Função de custo de transição para aprendizado justo}
  %\section{Correlação e preditores indiretos}
  %\section{Regularização}
  %\section{Ajuste de hiperparâmetros e matrizes de transição}

  %\chapter{Resultados e discussão}
  %\section{Significância para redes profundas}
  %\section{Estudo comparativo da função de custo de transição}
  %\section{Estudo comparativo com regularização}

  %\chapter{Conclusões}
  


  %\chapter{Conclusões}

  \backmatter
  \bibliographystyle{coppe-unsrt}
  \bibliography{thesis}

  %\appendix
  \include{appendix}
\end{document}
%% 
%%
%% End of file `example.tex'.
