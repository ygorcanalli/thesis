%%
%% This is file `example.tex',
%% generated with the docstrip utility.
%%
%% The original source files were:
%%
%% coppe.dtx  (with options: `example')
%% 
%% This is a sample monograph which illustrates the use of `coppe' document
%% class and `coppe-unsrt' BibTeX style.
%% 
%% \CheckSum{1620}
%% \CharacterTable
%%  {Upper-case    \A\B\C\D\E\F\G\H\I\J\K\L\M\N\O\P\Q\R\S\T\U\V\W\X\Y\Z
%%   Lower-case    \a\b\c\d\e\f\g\h\i\j\k\l\m\n\o\p\q\r\s\t\u\v\w\x\y\z
%%   Digits        \0\1\2\3\4\5\6\7\8\9
%%   Exclamation   \!     Double quote  \"     Hash (number) \#
%%   Dollar        \$     Percent       \%     Ampersand     \&
%%   Acute accent  \'     Left paren    \(     Right paren   \)
%%   Asterisk      \*     Plus          \+     Comma         \,
%%   Minus         \-     Point         \.     Solidus       \/
%%   Colon         \:     Semicolon     \;     Less than     \<
%%   Equals        \=     Greater than  \>     Question mark \?
%%   Commercial at \@     Left bracket  \[     Backslash     \\
%%   Right bracket \]     Circumflex    \^     Underscore    \_
%%   Grave accent  \`     Left brace    \{     Vertical bar  \|
%%   Right brace   \}     Tilde         \~}
%%
\documentclass[dsc,numbers]{coppe}
\usepackage{amsmath,amssymb}
\usepackage{hyperref}

\makelosymbols
\makeloabbreviations

\begin{document}
  \title{Técnicas de Ruído de Classe para Aprendizado de Máquina Justo}
  \foreigntitle{Thesis Title}
  \author{Ygor}{de Mello Canalli}
  \advisor{Prof.}{Geraldo}{Zimbrão da Silva}{D.Sc.}
  \advisor{Prof.}{Filipe}{Braida do Carmo}{D.Sc.}

  \examiner{Prof.}{Leandro Guimarães Marques Alvim}{D.Sc.}
  \examiner{Prof.}{Nome do Segundo Examinador Sobrenome}{Ph.D.}
  \examiner{Prof.}{Nome do Terceiro Examinador Sobrenome}{D.Sc.}
  \examiner{Prof.}{Nome do Quarto Examinador Sobrenome}{Ph.D.}
  \examiner{Prof.}{Nome do Quinto Examinador Sobrenome}{Ph.D.}
  \department{PESC}
  \date{02}{2011}

  \keyword{Aprendizado de Máquina Justo}
  \keyword{Ruído em Aprendizado de Máquina}
  \keyword{Terceira palavra-chave}

  \maketitle

  \frontmatter
  \dedication{A algu\'em cujo valor \'e digno desta dedicat\'oria.}

  \chapter*{Agradecimentos}

  Gostaria de agradecer a todos.

  \begin{abstract}

  Apresenta-se, nesta tese, uma abordagem ...

  \end{abstract}

  \begin{foreignabstract}

  In this work, we present ...

  \end{foreignabstract}

  \tableofcontents
  \listoffigures
  \listoftables
  \printlosymbols
  \printloabbreviations

  \mainmatter
  \chapter{Introdução}

  \chapter{Justiça em aprendizado de máquina}
  \section{Justiça, transparência, auditabilidade e privacidade}
  \section{Fontes de injustiça}
  \section{Definições e métricas}
  \section{Regulamentação}
  \section{Conjuntos de dados}
  \section{Abordagens para classificação justa}
  \section{Otimização multiobjetivo}

  \chapter{Ruído em aprendizado de máquina}
  \section{Definições e taxonomia de ruído}
  \section{Classificação na presença de ruído de classe}
  \section{Fatoração e correção de funções de custo}
  \section{Estimadores da matriz de transição}

  \chapter{Proposta}
  \section{Ruído e Injustiça}
  \section{Função de custo de transição para aprendizado justo}
  \section{Correlação e preditores indiretos}
  \section{Regularização}
  \section{Ajuste de hiperparâmetros e matrizes de transição}

  \chapter{Resultados e discussão}
  \section{Significância para redes profundas}
  \section{Estudo comparativo da função de custo de transição}
  \section{Estudo comparativo com regularização}

  \chapter{Conclusões}
  


  \chapter{Conclusões}

  \backmatter
  \bibliographystyle{coppe-unsrt}
  \bibliography{thesis}

  \appendix
  \chapter{Algumas Demonstra{\c c}\~oes}
\end{document}
%% 
%%
%% End of file `example.tex'.
